\Question{\currfilebase}
\begin{enumerate}
    \item Let this condition be \(C_1\). Since the game ends in the first round, the score must remain to be zero, and therefore
          \[
              \Prob(N = 0 \mid C_1) = 1,
          \]
          and for all other \(n \in \NN\) where \(n \neq 0\),
          \[
              \Prob(N = n \mid C_1) = 0.
          \]

          This means the p.g.f. for \(N\) conditional under \(C_1\) is just simply \(G(t \mid C_1) = \Prob(N = 0 \mid C_1) \cdot t^0 = 1\).

    \item Denote this condition be \(C_2\). Since in the first round, the game score does not change, and after the first round it is just as if this was a new game, so for all \(n \in \NN \cup \{0\}\), we must have
          \[
              \Prob(N = n \mid C_2) = \Prob(N = n),
          \]
          and hence
          \[
              G(t \mid C_2) = \sum_{n = 0}^{\infty} \Prob(N = n \mid C_2) \cdot x^n = \sum_{n = 0}^{\infty} \Prob(N = n) \cdot t^n = G(t).
          \]

    \item Denote the condition where the score is increased by \(1\) as \(C_3\). Since in the first round the game score increased by one, and after the first round it is just as if this was a new game, so for all \(n \in \NN\), we must have
          \[
              \Prob(N = n \mid C_3) = \Prob(N = n - 1),
          \]
          and
          \[
              \Prob(N = 0 \mid C_3) = 0.
          \]

          Hence,
          \[
              G(t \mid C_3) = \sum_{n = 0}^{\infty} \Prob(N = n \mid C_3) \cdot x^n = \sum_{n = 1}^{\infty} \Prob(N = n - 1) \cdot t^n = t \cdot \sum_{n = 0}^{\infty} \Prob(N = n) \cdot t^n= tG(t).
          \]

          Since in the first round, one of \(C_1, C_2\) and \(C_3\) must happen, we must have that
          \[
              G(t) = \Prob(C_1) \cdot G(t \mid C_1) + \Prob(C_2) \cdot G(t \mid C_2) + \Prob(C_3) \cdot G(t \mid C_3) = a + b G(t) + ctG(t).
          \]

          Hence, rearranging gives
          \[
              (1 - b - ct) G(t) = a,
          \]
          and hence
          \[
              G(t) = \frac{a}{(1 - b) - ct} = \frac{a / (1 - b)}{1 - ct / (1 - b)}
          \]

          Hence, using the infinite expansion, we have
          \begin{align*}
              G(t) & = \frac{a}{1-b} \cdot \sum_{k = 0}^{\infty} \left(\frac{ct}{1 - b}\right)^k   \\
                   & = \sum_{k = 0}^{\infty} \frac{a}{1 - b} \cdot \frac{c^k}{(1 - b)^k} \cdot t^k \\
                   & = \sum_{k = 0}^{\infty} \frac{ac^k}{(1 - b)^{k + 1}} \cdot t^k.
          \end{align*}

          But the coefficient before \(t^n\) is precisely the probability \(\Prob(N = n)\). This means
          \[
              \Prob(N = n) = \frac{ac^k}{(1 - b)^{k + 1}},
          \]
          as desired.

    \item We know that \(\mu = G'(1)\). We can find that
          \[
              G'(t) = \frac{ac}{[(1 - b) - ct]^2},
          \]
          and evaluating this at \(t = 1\) gives
          \[
              \mu = G'(1) = \frac{ac}{(1 - b - c)^2} = \frac{ac}{a^2} = \frac{c}{a}.
          \]

          Therefore, we have \(c = \mu a\)
          \begin{align*}
              \Prob(N = n) & = \frac{a c^k}{(a + c)^{k + 1}}                   \\
                           & = \frac{a (\mu a)^k}{(a + \mu a)^{k + 1}}         \\
                           & = \frac{a \mu^k a^k}{a^{k + 1} (1 + \mu)^{k + 1}} \\
                           & = \frac{\mu^k}{\mu^{k + 1}},
          \end{align*}
          as desired.
\end{enumerate}