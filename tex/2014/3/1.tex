\Question{\currfilebase}

Notice that
\[
    (1+ax)(1+bx)(1+cx) = 1 + (a + b + c) x + (ab + ac + bc) x^2 + abc x^3,
\]
and by comparing coefficients we have
\[
    q = bc + ca + ab, r = abc
\]

\begin{enumerate}
    \item Using the identities for the logarithms, we have
          \begin{align*}
              \ln(1 + qx^2 + rx^3) & = \ln(1 + ax) + \ln(1 + bx) + \ln(1 + cx)                                                                                                                         \\
                                   & = \sum_{k = 1}^{\infty} (-1)^{k + 1} \frac{(ax)^k}{k} + \sum_{k = 1}^{\infty} (-1)^{k + 1} \frac{(bx)^k}{k} + \sum_{k = 1}^{\infty} (-1)^{k + 1} \frac{(cx)^k}{k} \\
                                   & = \sum_{k = 1}^{\infty} (-1)^{k + 1} x^k \frac{a^k + b^k + c^k}{k},
          \end{align*}
          and hence
          \[
              S_k = \frac{a^k + b^k + c^k}{k},
          \]
          as desired.

    \item Since
          \begin{align*}
              S_2 & = \frac{a^2 + b^2 + c^2}{2}                 \\
                  & = \frac{(a + b + c)^2 - 2(ab + bc + ca)}{2} \\
                  & = \frac{0^2 - 2q}{2}                        \\
                  & = -q,
          \end{align*}
          \begin{align*}
              S_3 & = \frac{a^3 + b^3 + c^3}{3}                                    \\
                  & = \frac{(a + b + c)(a^2 + b^2 + c^2 - ab - bc - ca) + 3abc}{3} \\
                  & = abc                                                          \\
                  & = r,
          \end{align*}
          and
          \begin{align*}
              S_5 & = \frac{a^5 + b^5 + c^5}{5}                                                                        \\
                  & = \frac{(a^2 + b^2 + c^2)(a^3 + b^3 + c^3) - a^2b^2 (a + b) - a^2c^2 (a + c) - b^2 c^2 (b + c)}{5} \\
                  & = \frac{(-2q)(3r) + a^2 b^2 c + b^2 c^2 a + a^2 c^2 b}{5}                                          \\
                  & = \frac{-6qr + abc(ab + bc + ac)}{5}                                                               \\
                  & = \frac{-6qr + qr}{5}                                                                              \\
                  & = -qr.
          \end{align*}

          Therefore, \(S_2 S_3 = S_5\) as desired.

    \item Notice that
          \begin{align*}
              S_7 & = \frac{a^7 + b^7 + c^7}{7}                                                                      \\
                  & = \frac{(a^2 + b^2 + c^2)(a^5 + b^5 + c^5)}{7}                                                   \\
                  & = \frac{(-2q) \cdot (-5qr) - a^2 b^2 (a^3 + b^3) - b^2 c^2 (b^3 + c^3) - a^2 c^2 (a^3 + c^3)}{7} \\
                  & = \frac{10q^2 r - a^2 b^2 (3r - c^3) - b^2 c^2 (3r - a^3) - a^2 c^2 (3r - b^3)}{7}               \\
                  & = \frac{10q^2 r - 3r (a^2b^2 + b^2 c^2 + a^2 c^2) + a^2 b^2 c^2 (a + b + c)}{7}                  \\
                  & = \frac{10q^2 r - 3r\left[(ab + bc + ac)^2 - 2abc(a + b + c)\right] + r^2 \cdot 0}{7}            \\
                  & = \frac{10q^2 r - 3q^2r}{7}                                                                      \\
                  & = q^2r.
          \end{align*}

          Also, \(S_2 S_5 = (-q) \cdot (-qr) = q^2 r\), so \(S_2 S_5 = S_7\) as desired.

    \item Let \(a = 1, b = 1, c = -2\). \(q = bc + ca + ab = -3, r = -2\). This means \(S_2 = -q = 3, S_7 = q^2 r  = -18\). Notice that
          \[
              S_9 = \frac{a^9 + b^9 + c^9}{7} = \frac{1^9 + 1^9 + (-2)^9}{9} = - \frac{510}{9} = - \frac{170}{3},
          \]
          and this is obviously not \(S_2 S_7\) which gives a counterexample and the original statement is not true.
\end{enumerate}