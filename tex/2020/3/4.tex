\Question{\currfilebase}

We first show that \(Q\) lies on \(\Pi\). Notice that
\begin{align*}
    \vect{q} \cdot \vect{n} & = \left(\vect{x} - (\vect{x} \cdot \vect{n}) \vect{n}\right) \cdot \vect{n}     \\
                            & = \vect{x} \cdot \vect{n} - (\vect{x} \cdot \vect{n}) (\vect{n} \cdot \vect{n}) \\
                            & = \vect{x} \cdot \vect{n} - (\vect{x} \cdot \vect{n}) \abs*{\vect{n}}^2         \\
                            & = \vect{x} \cdot \vect{n} - \vect{x} \cdot \vect{n}                             \\
                            & = 0,
\end{align*}
so \(Q \in \Pi\).

To show \(PQ\) is perpendicular to \(\Pi\), we show that the vector \(\bvect{PQ}\) is parallel to the normal vector of \(\Pi\), and notice
\begin{align*}
    \bvect{PQ} & = \vect{q} - \vect{p}                  \\
               & = \vect{q} - \vect{x}                  \\
               & = - (\vect{x} \cdot \vect{n}) \vect{n}
\end{align*}
is a scalar multiple of the normal vector \(\vect{n}\), and so is parallel to \(\vect{n}\), and perpendicular to \(\Pi\).

\begin{enumerate}
    \item The normal vector to this plane is
          \[
              \vect{n} = \begin{pmatrix}
                  a \\
                  b \\
                  c
              \end{pmatrix}.
          \]

          We first find the projection of \(\ihat\) on \(\Pi\), and the point's position vector is given by
          \[
              \ihat - (\ihat \cdot \vect{n}) \vect{n} = \ihat - a \vect{n}.
          \]

          Hence, the reflection of \(\ihat\) in \(\Pi\) will be the point with position vector
          \begin{align*}
              \ihat + 2 \left[\left(\ihat - a \vect{n}\right) - \ihat\right] & = \ihat - 2 \left[a \vect{n}\right] \\
                                                                             & = \begin{pmatrix}
                                                                                     1 - 2a^2 \\
                                                                                     -2ab     \\
                                                                                     -2ac
                                                                                 \end{pmatrix}                    \\
                                                                             & = \begin{pmatrix}
                                                                                     a^2 + b^2 + c^2 - 2a^2 \\
                                                                                     -2ab                   \\
                                                                                     -2ac
                                                                                 \end{pmatrix}            \\
                                                                             & = \begin{pmatrix}
                                                                                     b^2 + c^2 - a^2 \\
                                                                                     -2ab            \\
                                                                                     -2ac
                                                                                 \end{pmatrix},
          \end{align*}
          as desired.

          Similarly, the image of \(\jhat\) under \(T\) is
          \[
              \begin{pmatrix}
                  -2ab            \\
                  a^2 + c^2 - b^2 \\
                  -2bc
              \end{pmatrix},
          \]
          and the image of \(\khat\) under \(T\) is
          \[
              \begin{pmatrix}
                  -2ac \\
                  -2bc \\
                  a^2 + b^2 - c^2
              \end{pmatrix}.
          \]

          Hence, the matrix \(\vect{M}\) which represents \(T\) is given by
          \[
              \vect{M} = \begin{pmatrix}
                  b^2 + c^2 - a^2 & -2ab            & -2ac            \\
                  -2ab            & a^2 + c^2 - b^2 & -2bc            \\
                  -2ac            & -2bc            & a^2 + b^2 - c^2
              \end{pmatrix}.
          \]

    \item Since the elements on the diagonal of \(\vect{M}\) are given by \(1 - 2a^2, 1 - 2b^2, 1 - 2c^2\), we must have
          \[
              1 - 2a^2 = 0.64, 1 - 2b^2 = 0.36, 1 - 2c^2 = 0,
          \]
          and hence
          \[
              a^2 = \frac{9}{50}, b^2 = \frac{8}{25}, c^2 = \frac{1}{2},
          \]
          which gives
          \[
              a = \pm \frac{3}{5\sqrt{2}}, b = \pm \frac{2 \sqrt{2}}{5}, c = \pm \frac{1}{\sqrt{2}},
          \]
          here the \(\pm\) signs do not necessarily match up.

          Since \(-2ab > 0\), \(-2ac > 0\), \(-2bc < 0\), we have \(a\) and \(b\), \(a\) and \(c\) take different signs, and \(b\) and \(c\) take the same sign.

          Hence,
          \[
              (a, b, c) = \left(\pm \frac{3}{5 \sqrt{2}}, \mp \frac{2 \sqrt{2}}{5}, \mp \frac{1}{\sqrt{2}}\right).
          \]

          We verify these triples indeed satisfy the non-diagonal elements as well.

          The Cartesian equation of the plane is therefore given by
          \begin{align*}
              ax + by + cz                                                        & = 0  \\
              \frac{3}{5 \sqrt{2}}x - \frac{2 \sqrt{2}}{5}y - \frac{1}{\sqrt{2}}z & = 0  \\
              3x - 4y - 5z                                                        & = 0.
          \end{align*}

    \item The line has equation
          \[
              l: \vect{r} = \lambda \vect{n}, \lambda \in \RR.
          \]

          A rotation about a line through \(\pi\) is simply a 'perpendicular' reflection of the point about the line, i.e. the reflection in the point on the line such that the point and the original point is perpendicular to the line.

          Let the original point be \(P\) with position vector \(\vect{x}\), and let the new point be \(Q\) with position vector \(\lambda \vect{n}\), we must have
          \[
              \left(\lambda \vect{n} - \vect{x}\right) \cdot \vect{n} = \lambda \vect{n} \cdot \vect{n} - \vect{x} \cdot \vect{n} = 0,
          \]
          which means
          \[
              \lambda = \vect{x} \cdot \vect{n},
          \]
          and
          \[
              \vect{q} = (\vect{x} \cdot \vect{n}) \vect{n}.
          \]

          Hence, the image of \(P\) under this transformation is
          \begin{align*}
              \vect{p} + 2(\vect{q} - \vect{p}) & = 2 \vect{q} - \vect{p}                            \\
                                                & = 2 (\vect{x} \cdot \vect{n}) \vect{n} - \vect{x}.
          \end{align*}

          If \(\vect{x} = \ihat\), the image is
          \[
              2 a \vect{n} - \ihat = \begin{pmatrix}
                  2a^2 - 1 \\
                  2ab      \\
                  2ac
              \end{pmatrix} = \begin{pmatrix}
                  a^2 - b^2 - c^2 \\
                  2ab             \\
                  2ac
              \end{pmatrix}.
          \]

          Similarly, the image of \(\jhat\) under this transformation is
          \[
              \begin{pmatrix}
                  2ab             \\
                  b^2 - a^2 - c^2 \\
                  2bc
              \end{pmatrix},
          \]
          and the image of \(\khat\) under this transformation is
          \[
              \begin{pmatrix}
                  2ac \\
                  2bc \\
                  c^2 - a^2 - b^2
              \end{pmatrix}.
          \]

          Hence, the matrix which represents this transformation, \(\vect{N}\), is given by
          \[
              \vect{N} = \begin{pmatrix}
                  a^2 - b^2 - c^2 & 2ab             & 2ac              \\
                  2ab             & b^2 - a^2 - c^2 & 2bc              \\
                  2ac             & 2bc             & c^2 - a^2 - b^2.
              \end{pmatrix}
          \]

    \item Notice that since \(\vect{N} = -\vect{M}\), and \(\vect{M}\) by definition is self-inverse, we have
          \[
              \vect{N} \vect{M} = - \vect{M} \vect{M} = - \vect{M}^2 = - \vect{I},
          \]
          which is an enlargement of scale factor \((-1)\) with the centre being the origin.
\end{enumerate}