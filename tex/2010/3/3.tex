\Question{\currfilebase}

An \(n\)-th root of unity takes the form \(\exp(\frac{k}{n} \cdot 2\pi I)\) for \(k = 0, \ldots, n - 1\), and specially, it is a primitive \(n\)th root of unity, if and only if the fraction \(\frac{k}{n}\) is irreducible (being reducible is equivalent to it being another \(m\)th root of unity where \(0 < m < n\)), and this is equivalent to \(\gcd(k, n) = 1\).

The two primitive \(4\)th roots of unity are when \(k = 1\) or \(3\), which gives \(i\) and \(-i\) as the two primitive roots.

Hence,
\[
    C_4(x) = (x - i)(x + i) = x^2 + 1.
\]

\begin{enumerate}
    \item For \(n = 1\), \(k = 0\), and \(\gcd(0, 1) = 1\). So the only \(1\)st root of unity is primitive, and hence
          \[
              C_1(x) = x - 1.
          \]

          For \(n = 2\), \(k = 0\) or \(1\), and only \(\gcd(1, 2) = 1\). So the only primitive \(2\)nd root of unity is \(\exp\left(\frac{1}{2} \cdot 2\pi i\right) = -1\), and hence
          \[
              C_2(x) = x + 1.
          \]

          For \(n = 3\), \(k = 1\) or \(2\) gives \(\gcd(k, n) = 1\). Hence, the primitive \(3\)rd roots of unity are all \(3\)rd roots of unity apart from \(x = 1\). Hence,
          \[
              C_3(x) = \frac{x^3 - 1}{x - 1} = x^2 + x + 1.
          \]

          For \(n = 5\), \(k = 1, 2, 3, 4\) or \(5\) gives \(\gcd(k, n) = 1\). Hence, the primitive \(5\)th roots of unity are all \(5\)th roots of unity apart from \(x = 1\). Hence,
          \[
              C_5(x) = \frac{x^5 - 1}{x - 1} = x^4 + x^3 + x^2 + x + 1.
          \]

          For \(n = 6\), \(k = 1\) or \(5\) gives \(\gcd(k, n) = 1\). Hence,
          \begin{align*}
              C_6 (x) & = \left(x - \exp\left(\frac{1}{6} \cdot 2\pi i\right)\right) \left(x - \exp\left(\frac{5}{6} \cdot 2\pi i\right)\right) \\
                      & = \left(x - \exp\left(\frac{1}{3} \cdot \pi i\right)\right) \left(x - \exp\left(- \frac{1}{3} \cdot \pi i\right)\right) \\
                      & = x^2 - 2 \cdot \cos \left(\frac{1}{3} \cdot \pi\right)x + 1                                                            \\
                      & = x^2 - x + 1.
          \end{align*}

    \item Notice that
          \begin{align*}
              x^4 + 1 & = (x^2 + i) (x^2 - i)                                                                                                                                                                                                                              \\
                      & = \left[x^2 - \exp\left(\frac{3}{4} \cdot 2 \pi i\right)\right] \left[x^2 - \exp\left(\frac{1}{4} \cdot 2 \pi i\right)\right]                                                                                                                      \\
                      & = \left[x - \exp\left(\frac{3}{8} \cdot 2 \pi i\right)\right] \left[x - \exp\left(\frac{7}{8} \cdot 2 \pi i\right)\right] \left[x - \exp\left(\frac{1}{8} \cdot 2 \pi i\right)\right] \left[x - \exp\left(\frac{5}{8} \cdot 2 \pi i\right)\right],
          \end{align*}
          and the roots to \(C_n(x)\) are
          \[
              \exp\left(\frac{1}{8} \cdot 2 \pi i\right), \exp\left(\frac{3}{8} \cdot 2 \pi i\right), \exp\left(\frac{5}{8} \cdot 2 \pi i\right), \exp\left(\frac{7}{8} \cdot 2 \pi i\right).
          \]

          Since the number on the denominator is \(8\) (and all fractions are reduced), we can conclude that if \(n\) exists, then \(n = 8\).

          On the other hand, for \(n = 8\), only \(k = 1, 3, 5\) and \(7\) give \(\gcd(k, n) = 1\). This means that \(n = 8\) satisfies that the primitive \(8\)-th roots of unity being
          \[
              \exp\left(\frac{1}{8} \cdot 2 \pi i\right), \exp\left(\frac{3}{8} \cdot 2 \pi i\right), \exp\left(\frac{5}{8} \cdot 2 \pi i\right), \exp\left(\frac{7}{8} \cdot 2 \pi i\right).
          \]

          Hence, \(n = 8\) satisfies \(C_n(x) = x^4 + 1\), and hence \(n = 8\).

    \item Since \(p\) is prime, for \(k = 1, 2, 3, \ldots, p - 1\), we must have \(\gcd(k, p) = 1\) (and for \(k = 0\), \(\gcd(k, p) = p \neq 1\)). This means that all the \(p\)th roots of unity apart from \(x = 1\) will be primitive \(p\)th roots of unity, and hence
          \[
              C_p(x) = \frac{x^p - 1}{x - 1} = 1 + x + x^2 + \cdots + x^{p - 1}.
          \]

    \item A root of \(C_q\) must take the form of
          \[
              \exp \left(\frac{Q}{q} \cdot 2\pi i\right)
          \]
          where \(0 \leq Q < q, \gcd(Q, q) = 1\).

          A root of \(C_r\) must take the form of
          \[
              \exp \left(\frac{R}{r} \cdot 2\pi i\right)
          \]
          where \(0 \leq R < r, \gcd(R, r) = 1\), and a root of \(C_s\) must take the form of
          \[
              \exp \left(\frac{S}{s} \cdot 2\pi i\right)
          \]
          where \(0 \leq S < s, \gcd(S, s) = 1\).

          Since a root to \(C_s\) must be a root to the right-hand side of the equation, and hence must be a root to the left-hand side of the equation, we have
          \[
              \exp \left(\frac{Q}{q} \cdot 2\pi i\right) = \exp \left(\frac{S}{s} \cdot 2\pi i\right).
          \]

          Since \(0 \leq \frac{Q}{q}, \frac{S}{s} < 1\), we must have
          \[
              \frac{Q}{q} = \frac{S}{s},
          \]
          and since they are both reduced fractions, we must have \(q = s\).

          Similarly, we also have \(q = r\).

          This means
          \[
              C_q(x) = C_q(x)^2,
          \]
          and hence
          \[
              C_q(x) (C_q(x) - 1) = 0.
          \]

          Since \(C_q\) is a polynomial, this means either \(C_q(x) = 0\) or \(C_q(x) = 1\), both of which are not possible given \(q\) is a positive integer. For the first case, this is impossible since this polynomial has infinitely many roots, but there are only finitely many \(q\)th roots of unity, and hence only finitely many primitive \(q\)th roots of unity.

          For the second case, this means that there is no primitive \(q\)th roots of unity. But for \(k = 1\), \(\gcd(k, q) = 1\), and hence there must be a primitive \(q\)th root of unity
          \[
              \exp \left(\frac{1}{q} \cdot 2\pi i\right),
          \]
          and this must be impossible.

          Hence, there are no positive integers \(q, r\) and \(s\) such that
          \[
              C_q(x) = C_r(x) \cdot C_s(x).
          \]
\end{enumerate}