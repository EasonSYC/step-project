\Question{\currfilebase}

Since \(P(x) = Q(x) R'(x) - Q'(x) R(x)\), we notice that
\[
    \frac{P(x)}{Q(x)^2} = \DiffOp{x} \frac{R(x)}{Q(x)}.
\]

Hence,
\[
    \int \frac{P(x)}{Q(x)^2} \Diff x = \frac{R(x)}{Q(x)} + C
\]
where \(C\) is a real constant.

\begin{enumerate}
    \item Since \(R(x) = a + bx + cx^2\), we have \(R'(x) = b + 2cx\). We let \(P(x) = 5x^2 - 4x - 3\) and \(Q(x) = 1 + 2x + 3x^2\), and hence \(Q'(x) = 6x + 2\).

          Hence,
          \[
              5x^2 - 4x - 3 = (1 + 2x + 3x^2)(b + 2cx) - (6x + 2)(a + bx + cx^2).
          \]

          Notice that
          \begin{align*}
              \RHS & = [6cx^3 + (3b + 4c) x^2 + (2b + 2c)x + b] - [6cx^3 + (6b + 2c)x^2 + (6a + 2b)x + 2a] \\
                   & = (-3b + 2c) x^2 + (-6a + 2c) + (-2a + b).
          \end{align*}

          Hence, we have
          \[
              \left\{
              \begin{aligned}
                  -3b + 2c                  & = 5,  \\
                  -6a + 2c = -4 \iff 3a - c & = 2,  \\
                  -2a + b                   & = -3.
              \end{aligned}
              \right.
          \]

          Notice that
          \[
              1 \cdot (-3b + 2c) + 2 \cdot (3a - c) + 3 \cdot (-2a + b) = 0,
          \]
          and
          \[
              1 \cdot 5 + 2 \cdot 2 - 3 \cdot 3 = 0,
          \]
          which means that these three equations are linearly dependent. Hence, let \(a = 0\), and hence \(b = -3\), \(c = -2\), \(R(x) = -3 x - 2x^2\), which gives
          \[
              \int \frac{5x^2 - 4x - 3}{(1 + 2x + 3x^2)^2} \Diff x = \frac{- 3x - 2x^2}{1 + 2x + 3x^2} + C_1.
          \]

          Letting \(a = 1\), and hence \(b = -1\), \(c = 1\), \(R(x) = 1 - x + x^2\), which gives
          \[
              \int \frac{5x^2 - 4x - 3}{(1 + 2x + 3x^2)^2} \Diff x = \frac{1 - x + x^2}{1 + 2x + 3x^2} + C_2.
          \]

          Notice that
          \[
              \frac{1 - x + x^2}{1 + 2x + 3x^2} - \frac{- 3x - 2x^2}{1 + 2x + 3x^2} = \frac{1 + 2x + 3x^2}{1 + 2x + 3x^2} = 1,
          \]
          and the integrals just differ by a constant. Different choices of \((a, b, c)\) lead to results which only differ by a constant.

    \item The differential equation we are attempting to solve is equivalent to
          \[
              \DiffFrac{y}{x} + \frac{\sin x - 2 \cos x}{1 + \cos x + 2 \sin x} y = \frac{5 - 3 \cos x + 4 \sin x}{1 + \cos x + 2 \sin x}.
          \]

          The integrating factor is
          \begin{align*}
              I(x) & = \exp \int \frac{\sin x - 2 \cos x}{1 + \cos x + 2 \sin x} \Diff x       \\
                   & = \exp \int - \frac{\Diff (1 + \cos x + 2 \sin x)}{1 + \cos x + 2 \sin x} \\
                   & = \exp (- \ln |1 + \cos x + 2 \sin x|)                                    \\
                   & = \frac{1}{1 + \cos x + 2 \sin x},
          \end{align*}
          and hence
          \[
              \DiffOp{x} \frac{y}{1 + \cos x + 2 \sin x} = \frac{5 - 3 \cos x + 4 \sin x}{(1 + \cos x + 2 \sin x)^2}.
          \]

          Let \(Q(x) = 1 + \cos x + 2 \sin x\), and let \(P(x) = 5 - 3 \cos x + 4 \sin x\). We have \(Q'(x) = 2 \cos x - \sin x\)

          Set \(R(x) = a + b \sin x + c \cos x\) for some real constant \(a, b\) and \(c\). We have \(R'(x) = b \cos x - c \sin x\). Hence,
          \[
              5 - 3 \cos x + 4 \sin x = (1 + \cos x + 2 \sin x)(b \cos x - c \sin x) - (2 \cos x - \sin x)(a + b \sin x + c \cos x).
          \]

          We expand the brackets on the right-hand side, and we have
          \begin{align*}
              \RHS & = b \cos x - c \sin x + b \cos^2 x - c \cos x \sin x + 2b \sin x \cos x - 2c \sin^2 x             \\
                   & \phantom{=} - 2a \cos x - 2b\sin x \cos x - 2c \cos^2 x + a \sin x + b \sin^2 x + c \sin x \cos x \\
                   & = (a - c) \sin x + (b - 2a) \cos x + (b - 2c) \left(\sin^2 x + \cos^2 x\right)                    \\
                   & = (b - 2c) + (a - c) \sin x + (b - 2a) \cos x,
          \end{align*}
          and hence by comparing coefficients, we have
          \[
              \left\{
              \begin{aligned}
                  b - 2c  & = 5,  \\
                  a - c   & = 4,  \\
                  -2a + b & = -3.
              \end{aligned}
              \right.
          \]

          Notice that
          \[
              1 \cdot (b - 2c) + (-2) \cdot (a - c) + (-1) \cdot (-2a + b) = 0,
          \]
          and
          \[
              1 \cdot 5 + (-2) \cdot 4 + (-1) \cdot (-3) = 0,
          \]
          so the system of linear equations is linearly dependent. Hence, set \(a = 0\), and we have \(b = -3, c = -4\), and we have \(R(x)  = -3 \sin x - 4 \cos x\).

          Hence,
          \begin{align*}
              \int \frac{5 - 3 \cos x + 4 \sin x}{(1 + \cos x + 2 \sin x)^2} & = \int \frac{P(x)}{Q(x)^2} \Diff x                         \\
                                                                             & = \frac{R(x)}{Q(x)} + C                                    \\
                                                                             & = - \frac{3 \sin x + 4 \cos x}{1 + \cos x + 2 \sin x} + C,
          \end{align*}
          and hence
          \[
              \frac{y}{1 + \cos x + 2 \sin x} = - \frac{3 \sin x + 4 \cos x}{1 + \cos x + 2 \sin x} + C,
          \]
          which means the general solution to the differential equation is
          \[
              y = - (3 \sin x + 4 \cos x) + C(1 + \cos x + 2 \sin x).
          \]
\end{enumerate}