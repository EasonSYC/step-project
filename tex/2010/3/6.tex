\Question{\currfilebase}

\begin{enumerate}
    \item The coordinates of \(P_1\) are
          \[
              P_1 (\cos \phi, \sin \phi, 0),
          \]
          and the coordinates of \(Q_1\) are
          \[
              Q_1 (- \sin \phi, \cos \phi, 0).
          \]

          Since the rotation is about \(z\)-axis, the position of \(R\) remains unchanged
          \[
              R_1 (0, 0, 1).
          \]

    \item This rotation axis is precisely \(OQ_1\), since it is contained in the \(x\)-\(y\) plane, and is perpendicular to \(OP_1\). Hence, the position of \(Q\) remains unchanged, and hence
          \[
              Q_2 (- \sin \phi, \cos \phi, 0).
          \]

          If we drop a perpendicular from \(P_2\) to the line \(OP_1\), and call the intersection be \(P'\). We can see from trigonometry that
          \[
              P_2 P' = \sin \lambda,
          \]
          and
          \[
              O P' = \cos \lambda.
          \]

          Hence, the \(x\)-coordinate of \(P_2\) is \(\cos \lambda \cos \phi\), and the \(y\)-coordinate of \(P_2\) is \(\cos \lambda \sin \phi\). The \(z\)-coordinate of \(P_2\) is \(\sin \lambda\), and hence
          \[
              P_2 (\cos \phi \cos \lambda, \sin \phi \cos \lambda, \sin \lambda).
          \]

          The relative positions of \(P, Q\) and \(R\) remains unchanged under rotation, and hence
          \begin{align*}
              \vect{r}_{R_2} & = \vect{r}_{P_2} \times \vect{r}_{Q_2}                                           \\
                             & = \begin{pmatrix}
                                     \cos \phi \cos \lambda \\
                                     \sin \phi \cos \lambda \\
                                     \sin \lambda
                                 \end{pmatrix} \times \begin{pmatrix}
                                                          - \sin \phi \\
                                                          \cos \phi   \\
                                                          0
                                                      \end{pmatrix}                                            \\
                             & = \begin{vmatrix}
                                     \ihat                  & \jhat                  & \khat        \\
                                     \cos \phi \cos \lambda & \sin \phi \cos \lambda & \sin \lambda \\
                                     - \sin \phi            & \cos \phi              & 0
                                 \end{vmatrix}                 \\
                             & = \begin{pmatrix}
                                     \sin \phi \cos \lambda \cdot 0 - \sin \lambda \cos \phi           \\
                                     - (\cos \phi \cos \lambda \cdot 0 + \sin \lambda \cdot \sin \phi) \\
                                     \cos \phi \cos \lambda \cdot \cos \phi + \sin \phi \cos \lambda \cdot \sin \phi
                                 \end{pmatrix} \\
                             & = \begin{pmatrix}
                                     - \sin \lambda \cos \phi  \\
                                     -  \sin \lambda \sin \phi \\
                                     \cos^2 \phi \cos \lambda  + \sin^2 \phi \cos \lambda
                                 \end{pmatrix}                            \\
                             & = \begin{pmatrix}
                                     - \sin \lambda \cos \phi  \\
                                     -  \sin \lambda \sin \phi \\
                                     \cos \lambda
                                 \end{pmatrix},
          \end{align*}
          and hence
          \[
              R_2 (- \sin \lambda \cos \phi, - \sin \lambda \sin \phi, \cos \lambda).
          \]

    \item The angle of rotation is the angle between \(O P_0\) and \(O P_2\), and hence
          \begin{align*}
              \cos \theta & = \frac{\bvect{O P_0} \cdot \bvect{O P_2}}{\abs*{\bvect{O P_0}} \cdot \abs*{\bvect{O P_2}}} \\
                          & = \begin{pmatrix}
                                  1 \\
                                  0 \\
                                  0
                              \end{pmatrix} \cdot \begin{pmatrix}
                                                      \cos \phi \cos \lambda \\
                                                      \sin \phi \cos \lambda \\
                                                      \sin \lambda
                                                  \end{pmatrix}                                                \\
                          & = \cos \phi \cos \lambda,
          \end{align*}
          as desired.

          The axis of this rotation must be perpendicular to both \(O P_1\) and \(O P_2\), and hence their cross product
          \begin{align*}
              \bvect{O P_0} \times \bvect{O P_2} & = \begin{pmatrix}
                                                         1 \\
                                                         0 \\
                                                         0
                                                     \end{pmatrix} \times \begin{pmatrix}
                                                                              \cos \phi \cos \lambda \\
                                                                              \sin \phi \cos \lambda \\
                                                                              \sin \lambda
                                                                          \end{pmatrix}                    \\
                                                 & = \begin{vmatrix}
                                                         \ihat                  & \jhat                  & \khat        \\
                                                         1                      & 0                      & 0            \\
                                                         \cos \phi \cos \lambda & \sin \phi \cos \lambda & \sin \lambda
                                                     \end{vmatrix} \\
                                                 & = \begin{pmatrix}
                                                         0              \\
                                                         - \sin \lambda \\
                                                         \sin \phi \cos \lambda
                                                     \end{pmatrix}
          \end{align*}
          is a vector in the direction of the axis.
\end{enumerate}