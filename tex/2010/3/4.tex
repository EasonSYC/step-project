\Question{\currfilebase}

\begin{enumerate}
    \item Since \(\alpha\) is a common root of both equations, we have
          \[
              \alpha^2 + a \alpha + b = 0, \alpha^2 + c \alpha + d = 0.
          \]

          Since \(0 = 0\), we have
          \begin{align*}
              \alpha^2 + a \alpha + b & = \alpha^2 + c \alpha + d \\
              a \alpha + b            & = c \alpha + d            \\
              (a - c) \alpha          & = - (b - d)               \\
              \alpha                  & = - \frac{b - d}{a - c},
          \end{align*}
          given that \(a \neq c\).

          We first prove the only-if direction of the statement. Putting this back to the first original equation, we have
          \[
              \left(- \frac{b - d}{a - c}\right)^2 + a \cdot \left(- \frac{b - d}{a - c}\right) + b = 0,
          \]
          and hence multiplying both sides by \((a - c)^2\), we get
          \[
              (b - d)^2 - a (b - d) (a - c) + b (a - c)^2 = 0,
          \]
          as desired.

          The if direction of the statement is as follows. Given this equation, dividing both sides by \((a - c)^2\) gives
          \[
              \left(- \frac{b - d}{a - c}\right)^2 + a \left(- \frac{b - d}{a - c}\right) + b = 0,
          \]
          and putting \(x = - \frac{b - d}{a - c}\) into the second equation gives
          \begin{align*}
              x^2 + cx + d & = \left(- \frac{b - d}{a - c}\right)^2 + c \left(- \frac{b - d}{a - c}\right) + d                                      \\
                           & = \frac{1}{(a - c)^2} \left[(b - d)^2 - c (b - d) (a - c) + d (a - c)^2\right]                                         \\
                           & = \frac{1}{(a - c)^2} \left[(b - d)^2 - a (b - d) (a - c) + a (b - d) (a - c) - c (b - d) (a - c) + d (a - c)^2\right] \\
                           & = \frac{1}{(a - c)^2} \left[(b - d)^2 - a (b - d) (a - c) + (b - d) (a - c)^2 + d (a - c)^2\right]                     \\
                           & = \frac{1}{(a - c)^2} \left[(b - d)^2 - a (b - d) (a - c) + b (a - c)^2\right]                                         \\
                           & = 0.
          \end{align*}

          This still holds if \(a \neq c\). For the only-if direction, we still have \((a - c) \alpha = -(b - d)\), and hence \((a - c)^2 \alpha^2 = (b - d)^2\). Putting \(\alpha\) into the first equation, and multiplying both sides by \((a - c)^2\) gives us
          \[
              (a - c)^2 \alpha^2 + a (a - c) \alpha (a - c) + b (a - c)^2 = 0,
          \]
          and hence
          \[
              (b - d)^2 - a (b - d) (a - c) + b (a - c)^2 = 0.
          \]

          For the if-direction, if \(a = c\), then \((b - d)^2 = 0\) and hence \(b = d\). This means the two quadratic equations are identical, which naturally leads to at least one common root.

    \item We first show that the original two equations have at least one common root if and only if \(x^2 + ax + b = 0\) and \(x^2 + (q - b)x + r = 0\) share a root.

          The only-if direction is as follows. Multiplying both sides of the first equation by \(x\), we get
          \[
              x^3 + ax^2 + bx = 0,
          \]
          and hence subtracting this from the second equation gives us
          \[
              x^2 + (q - b)x + r = 0,
          \]
          which means the common root of the original two equations must be a root to the new equation as well.

          For the if direction, multiplying both sides of \(x^2 + ax + b = 0\) by \(x\) and adding this to the new equation gives us that
          \[
              x^3 + (a + 1) x^2 + qx + r = 0.
          \]

          This means the common root of \(x^2 + ax + b = 0\) and \(x^2 + (q - b)x + r = 0\) must be a root to the cubic equation as well.

          Now, the equations \(x^2 + ax + b = 0\) and \(x^2 + (q - b)x + r = 0\) share a root, if and only if
          \[
              (b - r)^2 - a (b - r) (a - (q - b)) + b (a - (q - b))^2 = 0,
          \]
          which is equivalent to
          \[
              (b - r)^2 - a (b - r) (a + b - q) + b (a + b - q)^2 = 0.
          \]

          The two equations are equivalent to
          \[
              x^2 + \frac{5}{2}x + b = 0, x^3 + \frac{7}{2}x + \frac{5}{2}x + \frac{1}{2} = 0.
          \]

          Let \(a = \frac{5}{2}, b = b, q = \frac{5}{2}, r = \frac{1}{2}\), and the two equations have at least common root if and only if
          \[
              \left(b - \frac{1}{2}\right)^2 - \frac{5}{2} \left(b - \frac{1}{2}\right) \left(\frac{5}{2} + b - \frac{5}{2}\right) + b \left(\frac{5}{2} + b - \frac{5}{2}\right)^2 = 0.
          \]

          This simplifies to
          \[
              (2b - 1)^2 - 5 (2b - 1) b + 4 b^3 = 0,
          \]
          which is equivalent to
          \[
              4b^3 - 6b^2 + b + 1 = 0.
          \]

          Notice that
          \[
              4b^3 - 6b^2 + b + 1 = (b - 1)(4b^2 - 2b - 1),
          \]
          and hence
          \[
              b_1 = 1, b_{2, 3} = \frac{2 \pm \sqrt{2^2 + 4 \cdot 4}}{2 \cdot 4} = \frac{1 \pm \sqrt{5}}{4}.
          \]


\end{enumerate}