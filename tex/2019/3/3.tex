\Question{\currfilebase}

\begin{enumerate}
    \item Since \(L_1\) is a line of invariant points, for each point \((x, y) \in L_1\), we have
          \[
              \begin{pmatrix}
                  a & b \\
                  c & d
              \end{pmatrix}
              \begin{pmatrix}
                  x \\
                  y
              \end{pmatrix}
              = \begin{pmatrix}
                  x \\
                  y
              \end{pmatrix},
          \]
          and hence
          \[
              ax + by = x, cx + dy = y.
          \]

          Hence,
          \[
              (1 - a) x = by, (1 - d)y = cx,
          \]
          and hence
          \[
              (1 - a) x (1 - d)y = bycx,
          \]
          which simplifies to
          \[
              [(a - 1)(d - 1) - bc] xy = 0.
          \]

          If the line \(L_1\) is the line \(x = 0\), then \(by = 0\) for all \(y\) and \(dy = y\) for all \(y\), giving \(d = 1\) and \(b = 0\). Hence, \((a - 1) (d - 1) - bc = 0\).

          Similarly, if the line \(L_1\) is the line \(y = 0\), then \(ax = x\) for all \(x\) and \(cx = 0\) for all \(y\), giving \(a = 1\) and \(c = 0\). Hence, \((a - 1) (d - 1) - bc = 0\).

          Otherwise, there must be a point \((x, y) \in L_1\) such that \(xy \neq 0\), which means \((a - 1) (d - 1) - bc = 0\).

          Hence, in all cases, we must have \((a - 1) (d - 1) = bc\) as desired.

          If \(L_1\) does not pass through the origin, then \(y = mx + k\) for some \(k \neq 0\), or \(x = k\) for some \(k \neq 0\).

          In the first case, we have
          \[
              ax + b(mx + k) = x,
          \]
          and hence
          \[
              (a + bm - 1)x + bk = 0
          \]
          for all \(x\), meaning \(a + bm - 1 = 0\) and \(bk = 0\).

          Similarly,
          \[
              cx + d(mx + k) = mx + k,
          \]
          and hence
          \[
              (c + dm - m)x + (d - 1)k = 0
          \]
          for all \(x\), meaning \(c + dm - m = 0\) and \((d - 1)k = 0\).

          Since \(k \neq 0\), \(bk = 0\) and \((d - 1)k = 0\) implies \(b = 0\) and \(d = 1\) respectively. Putting those back into the first corresponding equations, this solves to \(a = 1\) and \(c = 0\), which means
          \[
              \vect{A} = \begin{pmatrix}
                  1 &   \\
                    & 1
              \end{pmatrix}
              = \vect{I}_2.
          \]

          In the second case where \(x = k\) for some \(k \neq 0\), we have
          \[
              ak + by = k,
          \]
          and hence
          \[
              by + (a - 1)k = 0
          \]
          for all \(y\), meaning \(b = 0\) and \((a - 1)k = 0\).

          Similarly,
          \[
              ck + dy = y,
          \]
          and hence
          \[
              (d - 1)y + ck = 0
          \]
          for all \(y\), meaning \(d - 1 = 0\) and \(ck = 0\).

          Since \(k \neq 0\), \((a - 1)k = 0\) and \(ck = 0\) implies \(a = 1\) and \(c = 0\) respectively. Hence,
          \[
              \vect{A} = \begin{pmatrix}
                  1 & 0 \\
                  0 & 1
              \end{pmatrix}
              = \vect{I}_2.
          \]

          Therefore, \(L_1\) not passing through the origin must imply that \(\vect{A}\) is precisely the \(2\) by \(2\) identity matrix.

    \item If \((x, y)\) is an invariant point, we have
          \[
              (a - 1) x + by = 0, cx + (d - 1)y = 0.
          \]

          If \(b = 0\), then \((a - 1)(d - 1) = bc = 0\), and hence \(a = 1\) or \(d = 1\).

          In the case where \(a = 1\), the first equation is trivially true, and the second equation simplifies to
          \[
              cx + (d - 1)y = 0,
          \]
          and hence the line \(L: cx + (d - 1)y = 0\) is a line of invariant points.

          In the case where \(d = 1\), the original equation simplifies to
          \[
              (a - 1)x = 0, cx = 0,
          \]
          and hence the line \(L: x = 0\) is a line of invariant points.

          If \(b \neq 0\), we want to show that all points on the line \(L: (a - 1)x + by = 0\) satisfy the second equation. We multiply \((d - 1)\) on both sides of the equation, and hence
          \[
              (a - 1)(d - 1)x + b (d - 1)y = 0,
          \]
          which is
          \[
              bcx + b(d - 1)y = 0.
          \]

          Since \(b \neq 0\), we divide \(b\) on both sides, giving
          \[
              cx + (d - 1)y = 0,
          \]
          which is precisely the second equation. Hence, \(L: (a - 1)x + by = 0\) is a line of invariant points under this case.

    \item We have \(L_2: y = mx + k\), \(k \neq 0\), we therefore have
          \[
              \begin{pmatrix}
                  a & b \\
                  c & d
              \end{pmatrix}
              \begin{pmatrix}
                  x \\
                  mx + k
              \end{pmatrix}
              =
              \begin{pmatrix}
                  X \\
                  mX + k
              \end{pmatrix},
          \]
          and hence
          \[
              ax + b(mx + k) = X, cx + d(mx + k) = mX + k.
          \]

          Putting the first equation into the second one gives us
          \[
              cx + d(mx + k) = m(ax + b(mx + k)) + k,
          \]
          which simplifies to
          \[
              (c + dm - am - bm^2)x + (dk - mbk - k) = 0,
          \]
          which is
          \[
              (bm^2 + (a - d)m - c)x + (mb - d + 1)k = 0.
          \]

          Since this is true for all \(x\) and \(k \neq 0\), we must have
          \[
              bm^2 + (a - d)m - c = 0, bm - d + 1 = 0.
          \]

          If \(b = 0\), then
          \[
              (a - d)m = c, d - 1 = 0,
          \]
          and hence \(d = 1\), \((a - 1)m = c\), and
          \[
              (a - 1)(d - 1) = 0, bc = 0,
          \]
          which gives
          \[
              (a - 1)(d - 1) = bc.
          \]

          If \(b \neq 0\), the second of those equations solve to
          \[
              m = \frac{d - 1}{b},
          \]
          and putting this back into the first equation, we have
          \[
              b \cdot \frac{(d - 1)^2}{b^2} + \frac{(a - d)(d - 1)}{b} - c = 0,
          \]
          and multiplying both sides by \(b\) gives
          \[
              (d - 1)^2 + (a - d)(d - 1) = bc,
          \]
          and hence
          \[
              (a - 1)(d - 1) = bc.
          \]

          Therefore, in both cases, we have \((a - 1)(d - 1) = bc\), as desired.
\end{enumerate}