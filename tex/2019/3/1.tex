\Question{\currfilebase}

\begin{enumerate}
    \item When \(k = 1\),
          \[
              \dot{x} = -x - y, \dot{y} = x - y.
          \]

          Hence,
          \begin{align*}
              \ddot{x} & = - \dot{x} - \dot{y}           \\
                       & = - \dot{x} - (x - y)           \\
                       & = -\dot{x} - x + y              \\
                       & = -\dot{x} - x + (-x - \dot{x}) \\
                       & = -2\dot{x} - 2x,
          \end{align*}
          and this gives
          \[
              \ddot{x} + 2\dot{x} + 2x = 0.
          \]

          The auxiliary equation to this differential equation is
          \[
              \lambda^2 + 2\lambda + 2 = 0,
          \]
          which solves to
          \[
              \lambda = -1 \pm i.
          \]

          The general solution for \(x\) is hence
          \[
              x(t) = \exp(-t) \left(A \sin t + B \cos t\right).
          \]

          This means
          \begin{align*}
              \dot{x}(t) & = - \exp(-t) \left(A \sin t + B \cos t\right) + \exp(-t) \left(A \cos t - B \sin t\right) \\
                         & = -x(t) + \exp(-t) \left(A \cos t - B \sin t\right),
          \end{align*}
          and hence
          \[
              y(t) = - \exp(-t) \left(A \cos t - B \sin t\right) = \exp(-t) \left(B \sin t - A \cos t\right).
          \]

          When \(t = 0\), \(x = x(0) = B = 1\), \(y = y(0) = -A = 0\). Hence,
          \[
              x(t) = \exp(-t) \cos t, y(t) = \exp(-t) \sin t.
          \]

          The graph of \(y\) against \(t\) looks as follows:
          \begin{center}
              \input{\currfiledir 1-diag1}
          \end{center}

          \(y\) is greatest at the first stationary point of \(y\), as shown in the graph. Note that
          \[
              \dot{y} = x - y = \exp(-t) \left(\cos t - \sin t\right),
          \]
          and hence
          \[
              \dot{y} = 0 \iff \cos t = \sin t \iff \tan t = 1,
          \]
          and the smallest positive solution to this is \(t = \frac{\pi}{4}\). The coordinate of the point is hence
          \[
              (x, y) = \left(\exp\left(-\frac{\pi}{4}\right) \cdot \frac{\sqrt{2}}{2}, \exp\left(-\frac{\pi}{4}\right) \cdot \frac{\sqrt{2}}{2}\right).
          \]

          Similarly, the graph of \(x\) against \(t\) looks as follows:
          \begin{center}
              \input{\currfiledir 1-diag2}
          \end{center}

          \(x\) is smallest at the first stationary point of \(x\), as shown in the graph. Note that
          \[
              \dot{x} = -x - y = -\exp(-t) \left(\cos t + \sin t\right),
          \]
          and hence
          \[
              \dot{x} = 0 \iff \cos t = - \sin t \iff \tan t = -1,
          \]
          and the smallest positive solution to this is \(t = \frac{3\pi}{4}\). The coordinate of the point is hence
          \[
              (x, y) = \left(- \exp\left(-\frac{3\pi}{4}\right) \cdot \frac{\sqrt{2}}{2}, \exp\left(-\frac{3\pi}{4}\right) \cdot \frac{\sqrt{2}}{2}\right).
          \]

          Without the \(\exp(-t)\) factor, the \(x\)-\(y\) graph will simply be a circle, and with this factor, it will be a spiral with exponentially decreasing radius. This is the polar curve \(r = \exp(-\theta)\). Hence, the \(x\)-\(y\) graph looks as follows.

          \begin{center}
              \input{\currfiledir 1-diag3}
          \end{center}

    \item Since \(\dot{x} = -x\), we must have \(x(t) = A \exp(-t)\), and since \(x(0) = 1\), we have \(A = 1\) and \(x(t) = \exp(-t)\).

          We have
          \[
              \dot{y} = \exp(-t) - y,
          \]
          and hence
          \[
              \dot{y} + y = \exp(-t).
          \]

          Multiplying both sides by \(\exp(t)\), we have
          \[
              e^t \dot{y} + e^t y = 1,
          \]
          and hence
          \[
              \DiffFrac{ye^t}{t} = 1,
          \]
          which gives
          \[
              ye^t = t + B,
          \]
          and hence
          \[
              y = \exp(-t) (t + B).
          \]

          Since \(y = 0\) when \(t = 0\), we must have \(B = 0\), and hence
          \[
              y = t \exp(-t).
          \]

          Note that
          \[
            \DiffFrac{y}{t} = \exp(-t) - t \exp(-t),
          \]
          and hence \(\DiffFrac{y}{t} = 0\) when \(t = 1\), which is when
          \[
            (x, y) = \left(e^{-1}, e^{-1}\right).
          \]

          Note that
          \[
            \DiffFrac{x}{y} = - \exp(-t),
          \]
          and hence \(\DiffFrac{x}{t} = 0\) when \(t = 0\), which is when
          \[
            (x, y) = (1, 0),
          \]
          and the tangent to the curve at this point will be vertical.

          Hence, the graph will look as follows:
          \begin{center}
            \input{\currfiledir 1-diag4}
          \end{center}
\end{enumerate}