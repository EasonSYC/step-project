\Question{\currfilebase}

\begin{enumerate}
    \item Let \([S]\) denote the area (2-D case) or the volume (3-D case) of \(S\).

          Let \(l = AB = BC = CA\), and hence we have
          \[
              [\Delta ABC] = \frac{l \cdot 1}{2} = \frac{l}{2}.
          \]

          By trigonometry, we also have
          \[
              [\Delta ABC] = \frac{l^2 \sin \frac{\pi}{3}}{2} = \frac{\sqrt{3}}{4} l^2,
          \]
          and hence
          \[
              \frac{\sqrt{3}}{4} l^2 = \frac{l}{2} \iff l = \frac{2}{\sqrt{3}}.
          \]

          On the other hand, splitting up the triangle, we have
          \begin{align*}
              [\Delta ABC] & = [\Delta ABP] + [\Delta BCP] + [\Delta ACP]                               \\
                           & = \frac{AB \cdot x_1}{2} + \frac{BC \cdot x_2}{2} + \frac{AC \cdot x_3}{2} \\
                           & = \frac{l}{2} \left(x_1 + x_2 + x_3\right).
          \end{align*}

          Since \([\Delta ABC] = [\Delta ABC]\), we must have \(x_1 + x_2 + x_3 = 1\).

          Let the angle bisectors of \(\angle BAC, \angle ABC\) and \(\angle ACB\) meet at a point \(O\) (this point exists since triangle \(ABC\) is equilateral).

          For \(X_1 = \min(X_1, X_2, X_3)\), this happens if and only if \(P\) is closer to \(AB\) than \(BC\) (including the equal case, \(X_1 \leq X_2\)), and \(P\) is closer to \(AB\) than \(AC\) (including the equal case, \(X_1 \leq X_3\)). This means \(P\) must lie on the side containing point \(A\) relative to \(BO\) (inclusive), and on the side containing point \(B\) relative to \(AO\) (inclusive).

          Hence, \(P\) must lie on or inside triangle \(AOB\), as shown in the diagram below.

          Without loss of generality (since a triangle has order-\(3\) rotational symmetry, and the centre of symmetry is \(O\)), we only consider the case where
          \[
              X = X_1 = \min(X_1, X_2, X_3).
          \]

          This means \(P\) must lie on or inside triangle \(AOB\). Consider the cumulative distribution function of \(X_1\) under this condition. By the following diagram, for \(0 \leq x \leq \frac{1}{3}\), we must have
          \begin{align*}
              F(x) & = \Prob(X \leq x)                                                                                         \\
                   & \propto [\Delta ABO] - [\Delta ARQ]                                                                       \\
                   & = \frac{l \cdot \frac{1}{3}}{2} \cdot \left[1 - \left(\frac{\frac{1}{3} - x}{\frac{1}{3}}\right)^2\right] \\
                   & = \frac{\frac{2}{\sqrt{3}} \cdot \frac{1}{3}}{2} \cdot \left[1 - \left(1 - 3x\right)^2\right]             \\
                   & = \frac{1}{3\sqrt{3}} \cdot \left[6x - 9x^2\right]                                                        \\
                   & = \frac{2x - 3x^2}{\sqrt{3}}.
          \end{align*}

          The maximum of \(x\) is \(\frac{1}{3}\), and hence \(F\left(\frac{1}{3}\right) = 1\). This means the constant of proportionality, \(k\), must satisfy
          \[
              k = \frac{F\left(\frac{1}{3}\right)}{\SqEvalAt{\frac{2x - 3x^2}{\sqrt{3}}}{x = \frac{1}{3}}} = \frac{1}{\frac{1}{3\sqrt{3}}} = 3\sqrt{3},
          \]
          and hence
          \[
              F(x) = 3(2x - 3x^2).
          \]

          Therefore, the probability density function of \(X\) for \(0 \leq x \leq \frac{1}{3}\) must satisfy
          \[
              f(x) = 6 - 18x = 6(1 - 3x),
          \]
          and \(0\) everywhere else, i.e.
          \[
              f(x) = \begin{cases}
                  6(1 - 3x), & 0 \leq x \leq \frac{1}{3}, \\
                  0,         & \text{otherwise}.
              \end{cases}
          \]

          Hence, the expectation of \(X\) satisfies
          \begin{align*}
              \Expt(X) & = \int_{\RR} x f(x) \Diff x                                               \\
                       & = \int_{0}^{\frac{1}{3}} (6x - 18x^2) \Diff x                             \\
                       & = \left[3x^2 - 6x^3\right]_{0}^{\frac{1}{3}}                              \\
                       & = 3 \cdot \left(\frac{1}{3}\right)^2 - 6 \cdot \left(\frac{1}{3}\right)^3 \\
                       & = \frac{3}{9} - \frac{2}{9}                                               \\
                       & = \frac{1}{9}.
          \end{align*}

    \item Let the regular tetrahedron be \(ABCD\) and the centroid be \(O\). Let \(AB = BC = BD = DA = l\). By trigonometry, we have
          \[
              \frac{l^3}{6\sqrt{2}} = \frac{1}{3} \cdot \frac{\sqrt{3} l^2}{4} \cdot 1,
          \]
          and hence
          \[
              l = \frac{\sqrt{3}}{\sqrt{2}}.
          \]


          Let the perpendicular distances from \(P\) to the face \(BCD, ACD, ABD\) and \(ABC\) be \(Y_1, Y_2, Y_3\) and \(Y_4\) respectively, and let
          \[
              Y = \min(Y_1, Y_2, Y_3, Y_4).
          \]

          By similar arguments as before, \(Y_1 = \min(Y_1, Y_2, Y_3, Y_4)\) if and only if \(P\) is on or inside the tetrahedron \(BCDO\).

          Let \(G\) be the cumulative distribution function of \(Y_1\) under this condition. For \(0 \leq y \leq \frac{1}{4}\), we have
          \begin{align*}
              G(y) & = \Prob(Y \leq y)                                                                                         \\
                   & \propto [BCDO] \cdot \left[1 - \left(\frac{\frac{1}{4} - y}{\frac{1}{4}}\right)^3\right]                  \\
                   & = \frac{1}{3} \cdot \frac{\sqrt{3} l^2}{4} \cdot \frac{1}{4} \cdot \left[1 - \left(1 - 4y\right)^3\right] \\
                   & = \frac{1}{16\sqrt{3}}  \cdot \frac{3}{2} \cdot \left[12y - 48y^2 + 64y^3\right]                          \\
                   & = \frac{\sqrt{3}}{32} \cdot \left[12y - 48y^2 + 64y^3\right]                                              \\
                   & = \frac{\sqrt{3} \left(3y - 12y^2 + 16y^3\right)}{8}.
          \end{align*}

          Since the maximum of \(y\) is \(\frac{1}{4}\), we must have \(G\left(\frac{1}{4}\right) = 1\), and hence the constant of proportionality, \(k\), must satisfy
          \[
              k = \frac{G\left(\frac{1}{4}\right)}{\SqEvalAt{\frac{\sqrt{3} \left(3y - 12y^2 + 16y^3\right)}{8}}{y = \frac{1}{4}}} = \frac{1}{\frac{\sqrt{3}}{32}} = \frac{32}{\sqrt{3}}.
          \]

          Hence,
          \[
              G(y) = 4 \left(3y - 12y^2 + 16y^3\right),
          \]
          and the probability density function of \(Y\) must satisfy for \(0 \leq y \leq \frac{1}{4}\)
          \[
              g(y) = 4 \left(3 - 24y + 48y^2 \right) = 12 \left(1 - 8y + 16y^2\right).
          \]

          Hence,
          \begin{align*}
              \Expt(y) & = \int_{\RR} y g(y) \Diff y                                                                                      \\
                       & = \int_{0}^{\frac{1}{4}} 12 \left(y - 8y^2 + 16y^3\right) \Diff y                                                \\
                       & = \left[6y^2 - 32y^3 + 48y^4\right]_{0}^{\frac{1}{4}}                                                            \\
                       & = 6 \cdot \left(\frac{1}{4}\right)^2 - 32 \cdot \left(\frac{1}{4}\right)^3 + 48 \cdot \left(\frac{1}{4}\right)^4 \\
                       & = \frac{3}{8} - \frac{1}{2} + \frac{3}{16}                                                                       \\
                       & = \frac{3 \cdot 2 - 1 \cdot 8 + 3}{16}                                                                           \\
                       & = \frac{1}{16}.
          \end{align*}

\end{enumerate}