\Question{\currfilebase}

Since \(\dot{y} = -2(y - z)\), differentiating both sides with respect to \(t\) gives
\begin{align*}
    \ddot{y} & = -2 \dot{y} + 2 \dot{z}           \\
             & = -2 \dot{y} + 2 (- \dot{y} - 3 z) \\
             & = -4 \dot{y} - 6z                  \\
             & = -4 \dot{y} - 3(\dot{y} + 2y)     \\
             & = -7 \dot{y} - 6y,
\end{align*}
and hence
\[
    \ddot{y} + 7\dot{y} + 6y = 0.
\]

The auxiliary equation
\[
    \lambda^2 + 7 \lambda + 6 = 0
\]
gives roots
\[
    \lambda_{1} = -1, \lambda_{2} = -6,
\]
and hence
\[
    y = Ae^{-t} + Be^{-6t}.
\]

Hence,
\[
    \dot{y} = -Ae^{-t} - 6Be^{-6t},
\]
and therefore,
\begin{align*}
    z & = \frac{\dot{y} + 2y}{2}                                   \\
      & = \frac{(-Ae^{-t} - 6Be^{-6t}) + 2(Ae^{-t} + Be^{-6t})}{2} \\
      & = \frac{Ae^{-t} - 4Be^{-6t}}{2}                            \\
      & = \frac{1}{2}A e^{-t} - 2B e^{-6t}.
\end{align*}

This set of general solution
\[
    (y, z) = \left(Ae^{-t} + Be^{-6t}, \frac{1}{2}A e^{-t} - 2B e^{-6t}\right),
\]
is exactly what is desired.

\begin{enumerate}
    \item \(y(0) = 5\) and \(z(0) = 0\) gives the system of linear equations
          \[
              \left\{
              \begin{aligned}
                  A + B             & = 5, \\
                  \frac{1}{2}A - 2B & = 0.
              \end{aligned}
              \right.
          \]

          This solves to \((A, B) = (4, 1)\). Hence,
          \[
              z_1(t) = 2e^{-t} - 2e^{-6t}.
          \]

    \item \(z(0) = z(1) = c\) gives the system of linear equations
          \[
              \left\{
              \begin{aligned}
                  \frac{1}{2}A - 2B              & = c, \\
                  \frac{1}{2e}A - \frac{2}{e^6}B & = c,
              \end{aligned}
              \right.
              \implies
              \left\{
              \begin{aligned}
                  A - 4B     & = 2c,    \\
                  e^5 A - 4B & = 2e^6c.
              \end{aligned}
              \right.
          \]

          Hence,
          \[
              A = \frac{2c(e^6 - 1)}{e^5 - 1},
          \]
          and therefore
          \begin{align*}
              B & = \frac{A - 2c}{4}                                        \\
                & = \frac{\frac{2c(e^6 - 1)}{e^5 - 1} - 2c}{4}              \\
                & = \frac{c}{2} \cdot \frac{(e^6 - 1) - (e^5 - 1)}{e^5 - 1} \\
                & = \frac{c e^5 (e - 1)}{2(e^5 - 1)}.
          \end{align*}

          This gives
          \[
              z_2(t) = \frac{c(e^6 - 1)}{e^5 - 1} e^{-t} - \frac{ce^5 (e - 1)}{e^5 - 1} e^{-6t}.
          \]

    \item Notice that
          \begin{align*}
               & \phantom{=} \sum_{n = -\infty}^{0} z_1(t - n)                                                \\
               & = \sum_{n = -\infty}^{0} [2e^{-t + n} - 2e^{-6t + 6n}]                                       \\
               & = 2\sum_{n = 0}^{\infty} [e^{-t-n} - e^{-6t-6n}]                                             \\
               & = 2 \left[e^{-t} \sum_{n = 0}^{\infty} e^{-n} - e^{-6t} \sum_{n = 0}^{\infty} e^{-6n}\right] \\
               & = 2 \left[\frac{e^{-t}}{1 - e^{-1}} - \frac{e^{-6t}}{1 - e^{-6}}\right]                      \\
               & = \frac{2e}{e - 1} e^{-t} - \frac{2e^6}{e^6 - 1} e^{-6t}.
          \end{align*}

          Hence, \(c\) must be such that
          \[
              \left\{
              \begin{aligned}
                  \frac{c(e^6 - 1)}{e^5 - 1} & = \frac{2e}{e - 1},             \\
                  \frac{2e^6}{e^6 - 1}       & = \frac{ce^5 (e - 1)}{e^5 - 1}.
              \end{aligned}
              \right.
          \]

          Both solves to precisely
          \[
              c = \frac{2e(e^5 - 1)}{(e - 1)(e^6 - 1)},
          \]
          and hence
          \[
              z_2(t) = \sum_{n = -\infty}^{0} z_1(t - n)
          \]
          for this value of \(c\).
\end{enumerate}