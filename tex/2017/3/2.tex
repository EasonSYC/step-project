\Question{\currfilebase}

\begin{enumerate}
    \item Let the complex number representing \(R(P)\) be \(z'\). Therefore,
          \begin{align*}
              z' - a & = \exp(i\theta) (z - a),                   \\
              z'     & = z \exp(i\theta) + a (1 - \exp(i\theta)),
          \end{align*}
          as desired.

    \item Let the complex number representing \(SR(P)\) be \(z''\). Therefore,
          \begin{align*}
              z'' - b & = \exp(i\phi) (z' - b),                                                                            \\
              z''     & = z' \exp(i\phi) + b(1 - \exp(i\phi)),                                                             \\
              z''     & = \left[z \exp(i\theta) + a (1 - \exp(i\theta))\right] \exp(i\phi) + b(1 - \exp(i\phi)),           \\
              z''     & = z\exp(i\left(\theta + \phi\right)) + a (1 - \exp(i \theta)) \exp(i \phi) + b (1 - \exp(i \phi)). \\
          \end{align*}

          This will be an anti-clockwise rotation around \(c\) over an angle of \((\theta + \phi)\), where
          \[
              c \left[1 - \exp(i (\theta + \phi))\right] = a \exp(i \phi) - a\exp(i \left(\theta + \phi\right)) + b - b \exp(i \phi),\\
          \]

          If \(\theta + \phi = 2n\pi\) for some integer \(n \in \ZZ\), \(1 - \exp(i (\theta + \phi)) = 0\), therefore \(c\) cannot be determined.

          Multiplying both sides by \(\exp\left(-\frac{i(\theta + \phi)}{2}\right)\), we have
          \begin{align*}
               & \phantom{=} c \left[\exp\left(-\frac{i(\theta + \phi)}{2}\right) - \exp\left(\frac{i (\theta + \phi)}{2}\right)\right]                                                                                                 \\
               & = a \left[\exp\left(\frac{i(\phi - \theta)}{2}\right) - \exp\left(\frac{i(\theta + \phi)}{2}\right)\right] + b\left[\exp\left(-\frac{i(\theta + \phi)}{2}\right) - \exp\left(\frac{i(\phi - \theta)}{2}\right)\right],
          \end{align*}
          and hence
          \begin{align*}
              -2ci\sin\left(\frac{\theta + \phi}{2}\right) & = - 2 ai \exp\left(\frac{i\phi}{2}\right) \sin \left(\frac{\theta}{2}\right) - 2 bi \exp\left(-\frac{i\theta}{2}\right) \sin \left(\frac{\phi}{2}\right), \\
              c \sin\left(\frac{\theta + \phi}{2}\right)   & = a \exp\left(\frac{i\phi}{2}\right) \sin \left(\frac{\theta}{2}\right) + b \exp\left(-\frac{i\theta}{2}\right) \sin \left(\frac{\phi}{2}\right).
          \end{align*}

          If \(\theta + \phi = 2\pi\), we will have \(z'' = z + a \exp(i\phi) - a + b (1 - \exp(i\phi)) = z + (b-a) (1 - \exp(i\phi))\), which is a translation by \((b-a) (1 - \exp(i\phi))\).

    \item If \(RS = SR\), then we have
          \begin{align*}
              a (1 - \exp(i \theta)) \exp(i \phi) + b (1 - \exp(i \phi))       & =
              b (1 - \exp(i \phi)) \exp(i \theta) + a (1 - \exp(i \theta)),                                                                          \\
              a (-1 + \exp(i \phi) + \exp(i \theta) - \exp(i (\theta + \phi))) & = b (-1 + \exp(i \phi) + \exp(i \theta) - \exp(i (\theta + \phi))), \\
              (a-b) (1 - \exp(i \phi)) (1 - \exp(i \theta))                    & = 0.
          \end{align*}

          Therefore, \(a = b\), or \(\phi = 2 n \pi\), or \(\theta = 2 n \pi\), for some integer \(n \in \ZZ\).
\end{enumerate}