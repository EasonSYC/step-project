\Question{\currfilebase}

\begin{enumerate}
    \item Notice that \(a = e^{\ln a}\) and hence \(a^x = e^{x \ln a}\), \(a^{\frac{x}{\ln a}} = e^x\)we have
    \begin{align*}
        F(y) &= \exp \left(\frac{1}{y} \int_0^y \ln f(x) \Diff x\right)\\
        &= a^{\frac{1}{y \ln a} \cdot \int_0^y \ln f(x) \Diff x}\\
        &= a^{\frac{1}{y} \cdot \int_0^y \frac{\ln f(x)}{\ln a} \Diff x}\\
        &= a^{\frac{1}{y} \cdot \int_0^y \log_a f(x) \Diff x}
    \end{align*}
    as desired.

    \item We have
    \begin{align*}
        H(y) &= \exp \left(\frac{1}{y} \int_0^y \ln f(x) g(x) \Diff x\right)\\
        &= \exp \left[\frac{1}{y} \int_0^y \left(\ln f(x) + \ln g(x)\right) \Diff x\right]\\
        &= \exp \left[\frac{1}{y} \left(\int_0^y \ln f(x) \Diff x + \int_0^y \ln g(x) \Diff x\right)\right]\\
        &= \exp \left(\frac{1}{y} \int_0^y \ln f(x) \Diff x\right) \cdot \exp \left(\frac{1}{y} \int_0^y \ln g(x) \Diff x\right)\\
        &= F(y) \cdot G(y).
    \end{align*}

    \item Let \(f(x) = b^x\).
    \begin{align*}
        F(y) &= \exp \left(\frac{1}{y} \int_0^y \ln f(x) \Diff x\right)\\
        &= b^{\frac{1}{y} \int_0^y \log_b f(x) \Diff x}\\
        &= b^{\frac{1}{y} \int_0^y \log_b b^x \Diff x}\\
        &= b^{\frac{1}{y} \int_0^y x \Diff x}\\
        &= b^{\frac{1}{y} \cdot \frac{y^2}{2}}\\
        &= b^{\frac{y}{2}}\\
        &= \sqrt{b^y}.
    \end{align*}

    \item Since \(F(y) = \sqrt{f(y)}\), we notice that \(f(y) = F(y)^2 = \exp \left(\frac{2}{y} \int_0^y \ln f(x) \Diff x\right)\), and therefore \(\ln f(y) = \frac{2}{y} \int_0^y \ln f(x) \Diff x\).
    
    We substitute \(g(y) = \ln f(y)\), and therefore
    \[
        y g(y) = 2 \int_0^y g(y) \Diff x.
    \]

    Therefore, differentiating both sides with respect to \(y\) gives us
    \[
        yg'(y) + g(y) = 2 g(y),
    \]
    and therefore
    \[
        -g(y) + y g'(y) = 0.
    \]

    Multiplying \(y^{-2}\) on both sides gives us
    \[
        -y^{-2}g(y) + y^{-1} g'(y) = 0,
    \]
    and therefore
    \[
        \frac{\Diff}{\Diff y} \frac{g(y)}{y} = 0,
    \]
    and therefore
    \[
        \frac{g(y)}{y} = C \implies g(y) = Cy.
    \]

    Therefore, we have
    \begin{align*}
        f(y) &= \exp g(y)\\
        &= \exp (Cy)\\
        &= b^y
    \end{align*}
    if we substitute \(b = \exp(C) > 0\), and therefore \(f(x) = b^y\) as desired.

\end{enumerate}