\Question{\currfilebase}

We have
\begin{align*}
    V(x) & = \Expt[(X-x)^2]                  \\
         & = \Expt(X^2 - 2xX + x^2)          \\
         & = \Expt(X^2) - 2x\Expt(X) + x^2   \\
         & = \sigma^2 + \mu^2 - 2x\mu + x^2.
\end{align*}

Therefore, if \(Y = V(X)\), then
\begin{align*}
    \Expt(Y) & = \Expt(V(X))                                   \\
             & = \Expt(\sigma^2 + \mu^2 - 2X\mu + X^2)         \\
             & = \sigma^2 + \mu^2 - 2\mu \Expt(X) + \Expt(X^2) \\
             & = \sigma^2 + \mu^2 - 2\mu^2 + \mu^2 + \sigma^2  \\
             & = 2\sigma^2.
\end{align*}

Let \(X \sim U[0, 1]\), we have \(\mu = \Expt(X) = \frac{1}{2}\), and \(\sigma^2 = \Var(X) = \frac{1}{12}\). Therefore,
\begin{align*}
    V(x) & = \frac{1}{12} + \frac{1}{4} - x + x^2 \\
         & = x^2 - x + \frac{1}{3}.
\end{align*}

The c.d.f. of \(X\) is \(F\), defined as
\[
    \Prob(X \leq x) = F(x) =
    \begin{cases}
        0, & x \leq 0,     \\
        x, & 0 < x \leq 1, \\
        1, & 1 < x
    \end{cases}
\]

Let the c.d.f. of \(Y\) be \(G\), we have \(G(y) = \Prob(Y \leq y)\).

Since \(V([0, 1]) = \left[\frac{1}{12}, \frac{1}{3}\right]\), we must have \(G(y) = 0\) for \(y \leq \frac{1}{12}\) and \(G(y) = 1\) for \(y > \frac{1}{3}\).

For \(y \in \left(\frac{1}{12}, \frac{1}{3}\right]\), we have
\begin{align*}
    G(y) = \Prob(Y \leq y) & = \Prob(V(X) \leq y)                                                                                        \\
                           & = \Prob\left(\left(x - \frac{1}{2}\right)^2 + \frac{1}{12} \leq y\right)                                    \\
                           & = \Prob\left(\abs{x - \frac{1}{2}} \leq \sqrt{y - \frac{1}{12}}\right)                                      \\
                           & = \Prob\left(\frac{1}{2} - \sqrt{y - \frac{1}{12}} \leq x \leq \frac{1}{2} + \sqrt{y - \frac{1}{12}}\right) \\
                           & = F\left(\frac{1}{2} + \sqrt{y - \frac{1}{12}}\right) - F\left(\frac{1}{2} - \sqrt{y - \frac{1}{12}}\right) \\
                           & = \left(\frac{1}{2} + \sqrt{y - \frac{1}{12}}\right) - \left(\frac{1}{2} - \sqrt{y - \frac{1}{12}}\right)   \\
                           & = 2 \sqrt{y - \frac{1}{12}}.
\end{align*}

Therefore, the p.d.f. of \(y\), \(g\) satisfies that for \(y \in \left(\frac{1}{12}, \frac{1}{3}\right]\),
\[
    g(y) = G'(y) = \frac{1}{\sqrt{y - \frac{1}{12}}}
\]
and \(0\) everywhere else.

Hence, we have
\begin{align*}
    \Expt(Y) & = \int_{\RR} y f(y) \Diff y                                                                                                                \\
             & = \int_{\frac{1}{12}}^{\frac{1}{3}} \frac{y}{\sqrt{y - \frac{1}{12}}} \Diff y                                                              \\
             & = \int_{y = \frac{1}{12}}^{y = \frac{1}{3}} 2y \Diff \sqrt{y - \frac{1}{12}}                                                               \\
             & = \left[2y\sqrt{y - \frac{1}{12}}\right]_{\frac{1}{12}}^{\frac{1}{3}} - 2\int_{\frac{1}{12}}^{\frac{1}{3}} \sqrt{y - \frac{1}{12}} \Diff y \\
             & = \left[2y\sqrt{y - \frac{1}{12}} - \frac{4}{3} \left(y - \frac{1}{12}\right)^{\frac{3}{2}}\right]_{\frac{1}{12}}^{\frac{1}{3}}            \\
             & = 2 \cdot \frac{1}{3} \cdot \frac{1}{2} - \frac{4}{3} \cdot \frac{1}{8}                                                                    \\
             & = \frac{1}{6}.
\end{align*}

Also, \(2 \sigma^2 = 2 \cdot \frac{1}{12} = \frac{1}{6} = \Expt(Y)\), so the formula we derived holds in this case.