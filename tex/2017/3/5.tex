\Question{\currfilebase}

Since we have \(x = r \cos \theta\) and \(y = r \sin \theta\), and \(r = f(\theta)\), we have
\begin{align*}
    \DiffFrac{x}{\theta} &= \DiffFrac{r}{\theta} \cdot \cos \theta + r \cdot \DiffFrac{\cos \theta}{\theta}\\
    &= f'(\theta) \cos \theta - f(\theta) \sin \theta,
\end{align*}
and
\begin{align*}
    \DiffFrac{y}{\theta} &= \DiffFrac{r}{\theta} \cdot \sin \theta + r \cdot \DiffFrac{\sin \theta}{\theta}\\
    &= f'(\theta) \sin \theta + f(\theta) \cos \theta,
\end{align*}

Therefore,
\begin{align*}
    \DiffFrac{y}{x} &= \frac{\DiffFrac{y}{\theta}}{\DiffFrac{x}{\theta}}\\
    &= \frac{f'(\theta) \sin \theta + f(\theta) \cos \theta}{f'(\theta) \cos \theta - f(\theta) \sin \theta}\\
    &= \frac{f'(\theta) \tan \theta + f(\theta)}{f'(\theta) - f(\theta) \tan \theta}.
\end{align*}

For the two curves, we must have
\[
    \LEvalAt{\DiffFrac{y}{x}}{f} \cdot \LEvalAt{\DiffFrac{y}{x}}{g} = -1
\]
for them to meet at right angles. Therefore,
\begin{align*}
    \frac{f'(\theta) \tan \theta + f(\theta)}{f'(\theta) - f(\theta) \tan \theta} \cdot
    \frac{g'(\theta) \tan \theta + g(\theta)}{g'(\theta) - g(\theta) \tan \theta} &= -1\\
    \left(f'(\theta) \tan \theta + f(\theta)\right) \cdot \left(g'(\theta) \tan \theta + g(\theta)\right) &= -\left(f'(\theta) - f(\theta) \tan \theta\right) \cdot \left(g'(\theta) - g(\theta) \tan \theta\right)\\
    f'(\theta) g'(\theta) (1 + \tan^2\theta) + f(\theta) g(\theta) (1 + \tan^2\theta) &= 0\\
    f'(\theta) g'(\theta) + f(\theta) g(\theta) &= 0.
\end{align*}

We have \(f\left(-\frac{\pi}{2}\right) = 4\). Let
\[
    g_a(\theta) = a(1 + \sin \theta).
\]

Therefore,
\[
    g_a'(\theta) = a \cos \theta,
\]
and we have
\[
    f'(\theta) (a \cos \theta) + f(\theta) a (1 + \sin \theta) = 0,
\]
and therefore
\[
    \DiffFrac{f(\theta)}{\theta} \cos \theta = - f(\theta) (1 + \sin \theta).
\]

By separating variables we have
\[
    \frac{\Diff f(\theta)}{f(\theta)} = - \frac{\Diff \theta (1 + \sin \theta)}{\cos \theta}.
\]

Notice that
\[
    -\frac{1 + \sin \theta}{\cos \theta} = - \frac{(1 - \sin \theta) (1 + \sin \theta)}{(1 - \sin \theta) \cos \theta} = - \frac{\cos \theta}{1 - \sin \theta} = \frac{\cos \theta}{\sin \theta - 1},
\]
integrating both sides gives us
\[
    \ln f(\theta) = \ln \Abs{\sin \theta - 1} + C = \ln \left(1 - \sin \theta\right) + C,
\]
which gives
\[
    f(\theta) = A (1 - \sin \theta).
\]

Since \(f\left(-\frac{\pi}{2}\right) = 4\), we have \(2A = 4\) and \(A = 2\), therefore \(f(\theta) = 2 (1 - \sin \theta)\).

\begin{center}
    \input{\currfiledir 5-diag}
\end{center}