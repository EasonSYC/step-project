\Question{\currfilebase}

By Vieta's Theorem, from the quartic equation in \(x\), we have
\[
    \alpha\beta + \alpha\gamma + \alpha\delta + \beta\gamma + \beta\delta + \gamma\delta = q,
\]
and from the cubic equation in \(y\), we have
\[
    \left(\alpha\beta + \gamma\delta\right) + \left(\alpha\gamma + \beta\delta\right) + \left(\alpha\delta + \beta\gamma\right) = -A.
\]

Therefore, \(A = -q\).

\begin{enumerate}
    \item Since \((p, q, r, s) = (0, 3, -6, 10)\), the cubic equation is reduced to
          \[
              y^3 - 3y^2 - 10y + 84 = 0,
          \]
          and therefore
          \[
              (y - 2)(y - 7)(y + 6) = 0.
          \]
          Therefore, \(y_1 = 7, y_2 = 2, y_3 = -6\), and \(\alpha\beta + \gamma\delta = 7\).

    \item We have
          \begin{align*}
              (\alpha + \beta)(\gamma + \delta) & = \alpha\gamma + \alpha\delta + \beta\gamma + \beta\delta                                                                                     \\
                                                & = \left(\alpha\beta + \alpha\gamma + \alpha\delta + \beta\gamma + \beta\delta + \gamma\delta\right) - \left(\alpha\beta + \gamma\delta\right) \\
                                                & = q - 7                                                                                                                                       \\
                                                & = 3 - 7                                                                                                                                       \\
                                                & = -4.
          \end{align*}

          By Vieta's Theorem, we have \(\alpha\beta\gamma\delta = s = 10\). Therefore, \(\alpha\beta\) and \(\gamma\delta\) must be roots to the equation
          \[
              x^2 - 7x + 10 = 0.
          \]

          The two roots are \(x = 2\) and \(x = 5\), and therefore \(\alpha\beta = 5\).

    \item We have from the other root that \(\gamma\delta = 2\).

          We notice that \((\alpha + \beta) + (\gamma + \delta) = -p = 0\). Therefore, from part 2, \((\alpha + \beta)\) and \((\gamma + \delta)\) are roots to the equation
          \[
              x^2 - 4 = 0.
          \]

          This gives us \(\alpha + \beta = \pm 2\) and \(\gamma + \delta = \mp 2\).

          Using the value of \(r\) and Vieta's Theorem, we have
          \[
              \alpha\beta\gamma + \alpha\beta\delta + \alpha\gamma\delta + \beta\gamma\delta = -r = 6.
          \]

          Plugging in \(\alpha\beta = 5\) and \(\gamma\delta = 2\), we have
          \[
              5(\gamma + \delta) + 2(\alpha + \beta) = 6.
          \]

          Therefore, it must be the case that \(\alpha + \beta = -2\) and \(\gamma + \delta = 2\).

          Hence, using the values of \(\alpha\beta\) and \(\gamma\delta\), \(\alpha\) and \(\beta\) are solutions to the quadratic equation \(x^2 + 2x + 5 = 0\), and \(\gamma\) and \(\delta\) are solutions to the quadratic equation \(x^2 - 2x + 2 = 0\).

          Solving this gives us \(\alpha, \beta = -1 \pm 2i\) and \(\gamma, \delta = 1 \pm i\). The solutions to the original quartic equation is
          \[
              x_{1, 2} = -1 \pm 2i, x_{3, 4} = 1 \pm i.
          \]
\end{enumerate}