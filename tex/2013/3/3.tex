\Question{\currfilebase}

Since \(\vect{p}_1 + \vect{p}_2 + \vect{p}_3 + \vect{4}_4 = \vect{0}\), we must have
\begin{align*}
    0 & = \vect{0} \cdot \vect{0}                                                                                                             \\
      & = \left(\vect{p}_1 + \vect{p}_2 + \vect{p}_3 + \vect{4}_4\right) \cdot \left(\vect{p}_1 + \vect{p}_2 + \vect{p}_3 + \vect{4}_4\right) \\
      & = \sum_{i = 1}^{4} \vect{p}_i \cdot \vect{p}_i + 2 \sum_{i = 1}^{3} \sum_{j = i + 1}^{4} \vect{p}_i \cdot \vect{p}_j.
\end{align*}

Since \(P_i\) are on the unit sphere, we must have \(\vect{p}_i \cdot \vect{p}_i = 1\). By symmetry, for all \(i \neq j\), \[
    \vect{p}_i \cdot \vect{p}_j
\]
must be some real constant, say \(k\).

Hence,
\[
    0 = 4 \cdot 1 + 2 \cdot 6 \cdot k,
\]
which solves to
\[
    k = -\frac{1}{3},
\]
as desired.

\begin{enumerate}
    \item We have
          \begin{align*}
              \sum_{i = 1}^{4} (X P_i)^2 & = \sum_{i = 1}^{4} \left(\vect{p}_i - \vect{x}\right) \cdot \left(\vect{p}_i - \vect{x}\right)                       \\
                                         & = \sum_{i = 1}^{4} \left(\vect{p_i} \cdot \vect{p_i} - 2 \vect{x} \cdot \vect{p}_i + \vect{x} \cdot \vect{x}\right)  \\
                                         & = \sum_{i = 1}^{4} \vect{p_i} \cdot \vect{p_i} - 2 \vect{x} \cdot \sum_{i = 1}^{4} + 4 \cdot \vect{x} \cdot \vect{x} \\
                                         & = \sum_{i = 1}^{4} 1 - 2 \vect{x} \cdot \vect{0} + 4 \cdot 1                                                         \\
                                         & = 4 - 0 + 4                                                                                                          \\
                                         & = 8.
          \end{align*}

    \item Since \(P_1 (0, 0, 1)\) and \(P_2 (a, 0, b)\), we must have
          \[
              \vect{p}_1 = \begin{pmatrix}
                  0 \\ 0 \\ 1
              \end{pmatrix},
              \vect{p}_2 = \begin{pmatrix}
                  a \\ 0 \\ b
              \end{pmatrix},
          \]
          and hence
          \[
              \vect{p}_1 \cdot \vect{p}_2 = 0 \cdot a + 0 \cdot 0 + 1 \cdot b = b = - \frac{1}{3}.
          \]

          We must have
          \[
              \abs*{\vect{p}_2} = \sqrt{a^2 + 0^2 + b^2} = \sqrt{a^2 + b^2} = 1,
          \]
          which means
          \[
              a = \frac{2\sqrt{2}}{3},
          \]
          as desired.

          The \(z\)-component of \(\vect{p}_3\) and \(\vect{p}_4\) must also be \(-\frac{1}{3}\), due to the dot product with \(vect{p}_1\) being equal to the \(z\)-component must also be equal to \(-\frac{1}{3}\).

          Let
          \[
              \vect{p}_3 = \begin{pmatrix}
                  c \\ d \\ -\frac{1}{3}
              \end{pmatrix},
          \]
          then from \(\sum_{i = 1}^{4} \vect{p}_i = \vect{0}\), we have
          \[
              \vect{p}_4 = \begin{pmatrix}
                  - c - \frac{2\sqrt{2}}{3} \\ -d \\ -\frac{1}{3}
              \end{pmatrix}.
          \]

          Since \(\vect{p}_3 \cdot \vect{p}_2 = -\frac{1}{3}\), we have
          \[
              \frac{2\sqrt{2}}{3} \cdot c + 0 \cdot d + \left(-\frac{1}{3}\right) \cdot \left(-\frac{1}{3}\right) = -\frac{1}{3},
          \]
          and hence
          \[
              \frac{2\sqrt{2}}{3} c = -\frac{4}{9},
          \]
          which means
          \[
              6\sqrt{2} c = -4,
          \]
          and hence
          \[
              c = - \frac{4}{6\sqrt{2}} = - \frac{\sqrt{2}}{3}.
          \]

          Now, since \(\vect{p}_3 \cdot \vect{p}_4 = - \frac{1}{3}\), we have
          \[
              c \cdot \left(- c - \frac{2\sqrt{2}}{3}\right) + d \cdot (-d) + \left(-\frac{1}{3}\right) \cdot \left(-\frac{1}{3}\right) = - \frac{1}{3}.
          \]

          Therefore,
          \[
              \left(-\frac{\sqrt{2}}{3}\right) \cdot \left(- \frac{\sqrt{2}}{3}\right) - d^2 = -\frac{4}{9},
          \]
          and hence
          \[
              d^2 = \frac{2}{3},
          \]
          giving
          \[
              d = \pm \frac{\sqrt{2}}{\sqrt{3}}.
          \]

          Hence,
          \[
              P_3 \left(-\frac{\sqrt{2}}{3}, \pm \frac{\sqrt{2}}{\sqrt{3}}, -\frac{1}{3}\right), P_4 \left(-\frac{\sqrt{2}}{\sqrt{3}}, \mp \frac{\sqrt{2}}{3}, -\frac{1}{3}\right).
          \]

    \item We have
          \begin{align*}
              \sum_{i = 1}^{4} \left(X P_i\right)^4 & = \sum_{i = 1}^{4} \left[\left(\vect{p}_i - \vect{x}\right) \cdot \left(\vect{p}_i - \vect{x}\right)\right]^2         \\
                                                    & = \sum_{i = 1}^{4} \left(\vect{p}_i \cdot \vect{p}_i - 2 \vect{x} \cdot \vect{p}_i + \vect{x} \cdot \vect{x}\right)^2 \\
                                                    & = \sum_{i = 1}^{4} \left(1 + 1 - 2 \vect{x} \cdot \vect{p}_i\right)^2                                                 \\
                                                    & = \sum_{i = 1}^{4} \left(2 - 2 \vect{x} \cdot \vect{p}_i\right)^2                                                     \\
                                                    & = 4 \sum_{i = 1}^{4} \left(1 - \vect{x} \cdot \vect{p}_i\right)^2.
          \end{align*}

          Let \(X(x, y, z)\). We have
          \begin{align*}
              \sum_{i = 1}^{4} \left(X P_i\right)^4
               & = 4 \sum_{i = 1}^{4} \left(1 - \vect{x} \cdot \vect{p}_i\right)^2                                                                                                                                     \\
               & = 4 \left[ \left(1 -
                  \begin{pmatrix}
                          x \\ y \\ z
                      \end{pmatrix}
                  \cdot
                  \begin{pmatrix}
                          0 \\ 0 \\ 1
                      \end{pmatrix}
                  \right)^2 + \left(1 -
                  \begin{pmatrix}
                          x \\ y \\ z
                      \end{pmatrix}
                  \cdot
                  \begin{pmatrix}
                          \frac{2\sqrt{2}}{3} \\ 0 \\ -\frac{1}{3}
                      \end{pmatrix}
              \right)^2  \right.                                                                                                                                                                                       \\
               & \phantom{=} \left. + \left(1 -
                  \begin{pmatrix}
                          x \\ y \\ z
                      \end{pmatrix}
                  \cdot
                  \begin{pmatrix}
                          -\frac{\sqrt{2}}{3} \\ \frac{\sqrt{2}}{\sqrt{3}} \\ -\frac{1}{3}
                      \end{pmatrix}
                  \right)^2+ \left(1 -
                  \begin{pmatrix}
                          x \\ y \\ z
                      \end{pmatrix}
                  \cdot
              \begin{pmatrix}
                          -\frac{\sqrt{2}}{3} \\ -\frac{\sqrt{2}}{\sqrt{3}} \\ -\frac{1}{3}
                      \end{pmatrix} \right)^2\right]                                                                                               \\
               & = 4 \left[(1 - z)^2 + \left(1 - \frac{2\sqrt{2}}{3}x + \frac{1}{3}z\right)^2 \right.                                                                                                                  \\
               & \phantom{=} \left. + \left(1 + \frac{\sqrt{2}}{3}x - \frac{\sqrt{2}}{\sqrt{3}}y + \frac{1}{3}z\right)^2  + \left(1 + \frac{\sqrt{2}}{3}x + \frac{\sqrt{2}}{\sqrt{3}}y + \frac{1}{3}z\right)^2 \right] \\
               & = 4\left(4 + \frac{4}{3} x^2 + \frac{4}{3}y^2 + \frac{4}{3}z^2\right)                                                                                                                                 \\
               & = 4 \left[4 + \frac{4}{3}\right]                                                                                                                                                                      \\
               & = 4 \cdot \frac{16}{3}                                                                                                                                                                                \\
               & = \frac{64}{3}
          \end{align*}
          is a constant, independent of the position of \(X\).
\end{enumerate}