\Question{\currfilebase}

For \(\omega = \exp \frac{2 \pi i}{n}\), we have for \(k = 0, 1, 2, \ldots, n - 1\), that \(\omega^k = \exp \frac{2\pi i k}{n}\). Therefore,
\[
    (\omega^k)^n = \exp \frac{2\pi i k n}{n} = \exp (2\pi i k) = 1.
\]

Also, notice that \(\arg \omega^k = \frac{2k\pi}{n}\), which means that all \(\omega^k\)s are different.

This means that \(\omega^0 = 1, \omega^1 = 1, \omega^2, \ldots, \omega^{n - 1}\) are exactly the \(n\) roots to the polynomial \(z^n - 1\), which has leading coefficient 1.

Therefore, we must have
\[
    (z - 1)(z - \omega) \cdots (z - \omega^{n - 1}) = z^n - 1,
\]
as desired.

For the following parts, W.L.O.G. let the orientation of the polygon be such that \(X_k = \omega^k\).

\begin{enumerate}
    \item Let \(z\) represent the complex number for \(P\), we have
          \begin{align*}
              \prod_{k = 0}^{n - 1} \abs*{P X_k} & = \prod_{k = 0}^{n - 1} \abs*{z - \omega^k}   \\
                                                 & = \abs*{\prod_{k = 0}^{n - 1} (z - \omega^k)} \\
                                                 & = \abs*{z^n - 1}.
          \end{align*}

          Since \(P\) is equidistant from \(X_0\) and \(X_1\), we must have that \(P = r \exp\left(\frac{\pi i}{n}\right)\) for some \(r \in \RR\), where \(\abs*{r} = \abs*{OP}\). Therefore, we have
          \begin{align*}
              \prod_{k = 0}^{n - 1} \abs*{P X_k} & = \abs*{z^n - 1}                                  \\
                                                 & = \abs*{r^n \exp\left(\frac{\pi i}{2}\right) - 1} \\
                                                 & = \abs*{-r^n - 1}                                 \\
                                                 & = \abs*{r^n + 1}.
          \end{align*}

          If \(n\) is even, then \(r^n = \abs*{r}^n > 0\), and therefore \(\abs*{r^n + 1} = r^n + 1 = \abs*{r}^n + 1 = \abs*{OP}^n + 1\) as desired.

          If \(n\) is odd, and \(r > 0\), then \(r^n = \abs*{r}^n > 0\), and
          \begin{align*}
              \LHS & = \abs*{r^n + 1}   \\
                   & = r^n + 1          \\
                   & = \abs*{r}^n + 1   \\
                   & = \abs*{OP}^n + 1.
          \end{align*}

          When \(-1 \leq r < 0\), we have \(-1 \leq r^n = -\abs*{r}^n < 0\), and
          \begin{align*}
              \LHS & = \abs*{r^n + 1}    \\
                   & = r^n + 1           \\
                   & = -\abs*{r}^n + 1   \\
                   & = -\abs*{OP}^n + 1.
          \end{align*}

          When \(r < -1\), we have \(r^n = -\abs*{r}^n < -1\), and
          \begin{align*}
              \LHS & = \abs*{r^n + 1}   \\
                   & = -r^n - 1         \\
                   & = \abs*{r}^n - 1   \\
                   & = \abs*{OP}^n - 1.
          \end{align*}

          In summary, when \(n\) is odd, we have
          \[
              \prod_{k = 0}^{n - 1} \abs*{P X_k} =
              \begin{cases}
                  \abs*{OP}^n + 1,  & P \text{ is in the first quadrant},                       \\
                  -\abs*{OP}^n + 1, & P \text{ is in the third quadrant and } \abs*{OP} \leq 1, \\
                  \abs*{OP}^n - 1,  & P \text{ is in the third quadrant and } \abs*{OP} > 1.
              \end{cases}
          \]

    \item Notice that for a general point \(P\) whose complex number is \(z\), we have
          \begin{align*}
              \prod_{k = 1}^{n - 1} \abs*{P X_k} & = (z - \omega) (z - \omega^2) \cdots \left(z - \omega^{n - 1}\right) \\
                                                 & = \frac{z^n - 1}{z - 1}                                              \\
                                                 & = 1 + z + z^2 + \cdots + z^{n - 1}.
          \end{align*}

          If we let \(P = X_0\), \(z = 1\), and \(\RHS = n\), just as we desired.
\end{enumerate}