\Question{\currfilebase}

\begin{enumerate}
    \item For \(y^2 = 4ax\), we have \(x = \frac{y^2}{4a}\), and therefore
          \[
              \DiffFrac{x}{y} = \frac{2y}{4a}.
          \]

          Therefore, the normal through \(Q\), \(l_Q\) satisfies that
          \[
              l_Q: x - aq^2 = - \frac{4a}{2 \cdot 2aq} \cdot \left(y - 2aq\right),
          \]
          i.e.
          \[
              l_Q: q(x - aq^2) =  -\left(y - 2aq\right).
          \]

          Since \(P \in l_Q\), we must have
          \begin{align*}
              q(ap^2 - aq^2) & =  -\left(2ap - 2aq\right) \\
              aq(p+q)(p-q)   & = -2a (p-q)                \\
              pq + q^2       & = -2                       \\
              q^2 + pq + 2   & = 0
          \end{align*}
          as desired.

    \item We also have
          \[
              r^2 + pr + 2 = 0.
          \]
          Since \(q \neq r\), \(q, r\) are the solutions to the equation
          \[
              x^2 + px + 2 = 0,
          \]
          and therefore \(q + r = -p, qr = 2\).

          Note that the equation for \(QR\) satisfies that
          \[
              m_{QR} = \frac{2ar - 2aq}{ar^2 - aq^2} = \frac{2}{r + q}.
          \]

          Therefore, \(l_{QR}\) satisfies that
          \begin{align*}
              l_{QR}: y - 2aq & = \frac{2}{r + q} (x - aq^2)                                      \\
              y               & = \frac{2}{r + q} \left(x - aq^2 + \frac{r + q}{2}\cdot2aq\right) \\
              y               & = \frac{2}{r + q} \left(x - aq^2 + aq^2 + aqr\right)              \\
              y               & = \frac{2}{r + q} \left(x + aqr\right)                            \\
              y               & = -\frac{2}{p} (x + 2a).
          \end{align*}

          This passes through a fixed point \((-2a, 0)\).

    \item \(OP\) has equation \(y = \frac{2ap}{ap^2}x\), which is \(y = \frac{2x}{p}\).
          Therefore, since \(T = OP \cap QR\), \(x_T\) must satisfy that
          \begin{align*}
              -\frac{2}{p} (x + 2a) & = \frac{2x}{p}, \\
              -(x + 2a)             & = x             \\
              x                     & = -a.
          \end{align*}

          Therefore, \(y_T = -\frac{2a}{p}\), \(T\left(-a, -\frac{2a}{p}\right)\) lies on the line \(x = -a\) which is independent of \(p\).

          The distance from the \(x\)-axis to \(T\) is \(\Abs{\frac{2a}{p}} = \frac{2a}{\Abs{p}}\).

          Notice that since \(qr = 2\), \(q\) and \(r\) must take the same parity, and therefore \(\Abs{p} = \Abs{q} + \Abs{r}\). By the AM-GM inequality, we have
          \[
              \Abs{q} + \Abs{r} \geq 2 \sqrt{\Abs{q} \cdot \Abs{r}} = 2\sqrt{2},
          \]
          with the equal sign holding if and only if \(\Abs{q} = \Abs{r}\), \(q = r\), which is impossible.

          Therefore, \(\Abs{p} > 2\sqrt{2}\) and therefore \(\frac{2a}{\Abs{p}} < \sqrt{2}\) as desired.
\end{enumerate}