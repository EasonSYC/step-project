\Question{\currfilebase}

Notice that
\begin{align*}
    x^a (x^b (x^c y)' )' & = x^a (x^b (c x^{c - 1} y + x^c y'))'                                                                             \\
                         & = x^a \left[x^{b + c - 1} \left(cy + xy' \right)\right]'                                                          \\
                         & = x^a \left[(b + c - 1) x^{b + c - 2} \left(cy + xy' \right) + x^{b + c - 1}  \left(cy' + y' + xy''\right)\right] \\
                         & = x^{a + b + c - 2} \left[(b + c - 1) \left(cy + xy' \right) + x \left(cy' + y' + xy''\right)\right]              \\
                         & = x^{a + b + c - 2} \left[x^2 y'' + (b + 2c) xy' + (b + c - 1) cy\right].
\end{align*}

Comparing this with the left-hand side of the original equation, we must have
\[
    \left\{
    \begin{aligned}
        a + b + c - 2 & = 0,         \\
        b + 2c        & = 1 - 2p,    \\
        (b + c - 1)c  & = p^2 - q^2.
    \end{aligned}
    \right.
\]

The second equation gives
\[
    b = 1 - 2p - 2c,
\]
and putting this into the third equation gives
\[
    (1 - 2p - 2c + c - 1)c = p^2 - q^2,
\]
and hence
\[
    c^2 + 2pc + p^2 - q^2 = 0.
\]

This gives
\[
    (c + (p - q))(c + (p + q)) = 0,
\]
and hence
\[
    c_1 = -p + q, c_2 = -p - q.
\]

Putting this back, we get
\[
    b_1 = 1 - 2p - 2(-p + q) = 1 - 2q, b_2 = 1 - 2p - 2(-p - q) = 1 + 2q,
\]
and since \(a = 2 - b - c\) from the first equation, we have
\[
    a_1 = 2 - (1 - 2q) - (-p + q) = 1 + p + q
\]
and
\[
    a_2 = 2 - (1 + 2q) - (-p - q) = 1 + p - q
\]

Hence, the solutions are
\[
    \left\{
    \begin{aligned}
        a & = p \pm q + 1, \\
        b & = \mp 2q + 1,  \\
        c & = -p \pm q.    \\
    \end{aligned}
    \right.
\]

\begin{enumerate}
    \item In the case where \(f(x) = 0\). We must have
          \[
              x^{a} \left(x^{b} (x^{c} y)'\right)' = 0,
          \]
          and hence
          \[
              \left(x^{b} (x^{c} y)'\right)' = 0.
          \]

          Therefore, we must have by integration
          \[
              x^{b} (x^{c} y)' = C_1
          \]
          for some (real) constant \(C_1\).

          Hence,
          \[
              (x^{c} y)' = C_1 x^{-b}.
          \]

          There are two cases here:
          \begin{enumerate}
              \item When \(b = 1\) i.e. \(q = 0\), the right-hand side is \(C_1 x^{-1}\), and the left-hand side is \((x^{c} y)'\). Integrating both sides give
                    \[
                        x^{c} y = C_1 \ln x + C_2
                    \]
                    for some (real) constant \(C_2\).

                    Hence,
                    \[
                        y = x^{-c} (C_1 \ln x + C_2)
                    \]
                    for some (real) constants \(C_1, C_2\).

                    When \(q = 0\), \(c = -p\), and hence
                    \[
                        y = x^p (C_1 \ln x + C_2).
                    \]

              \item When \(b \neq 1\) i.e. \(q \neq 0\), integrating both sides give
                    \[
                        x^{c} y = \frac{C_1 x^{-b + 1}}{-b + 1} + C_2
                    \]
                    for some (real) constant \(C_2\).

                    Hence,
                    \[
                        y = x^{-c} \left(\frac{C_1 x^{-b + 1}}{-b + 1} + C_2\right)
                    \]
                    for some (real) constant \(C_1, C_2\).

                    Hence,
                    \begin{align*}
                        y & = x^{-(-p \pm q)} \left(\frac{C_1 x^{-(\mp 2q + 1) + 1}}{-(\mp 2q + 1) + 1} + C_2\right) \\
                          & = x^{p \mp q} \left(\frac{C_1 x^{\pm 2q}}{\pm 2q} + C_2\right).                          \\
                          & = \frac{C_1}{\pm 2q} x^{p \pm q} + C_2 x^{p \mp q}                                       \\
                          & = C_3 x^{p \pm q} + C_2 x^{p \mp q},
                    \end{align*}
                    for some (real) constant \(C_2, C_3\).
          \end{enumerate}

    \item This is when \(q = 0\) and \(f(x) = x^n\). We have \(a = p + 1, b = 1\) and \(c = -p\), and the original differential equation reduces to
          \[
              x^{p + 1} \left(x \left(x^{-p} y\right)' \right)' = x^n,
          \]
          and hence
          \[
              \left(x \left(x^{-p} y\right)' \right)' = x^{n - p - 1}.
          \]

          There are two cases here:
          \begin{enumerate}
              \item If \(n - p - 1 = -1\), i.e. \(n = p\), we have, by integration,
                    \[
                        x \left(x^{-p} y\right)' = \ln x + C_1.
                    \]

                    This gives
                    \[
                        \left(x^{-p} y\right)' = \frac{\ln x}{x} + \frac{C_1}{x},
                    \]
                    and hence by integration
                    \[
                        x^{-p} y = \frac{(\ln x)^2}{2} + C_1 \ln x + C_2.
                    \]

                    This solves to
                    \[
                        y = \frac{x^p (\ln x)^2}{2} + C_1 x^p \ln x + C_2 x^p.
                    \]

              \item If \(n - p - 1 \neq -1\), i.e. \(n \neq p\), we have
                    \[
                        x \left(x^{-p} y\right)' = \frac{x^{n - p}}{n - p} + C_1.
                    \]

                    This gives
                    \[
                        \left(x^{-p} y\right)' = \frac{x^{n - p - 1}}{n - p} + \frac{C_1}{x}.
                    \]

                    Since \(n - p - 1 \neq -1\), by integration we have
                    \[
                        x^{-p} y = \frac{x^{n - p}}{(n - p)^2} + C_1 \ln x + C_2,
                    \]
                    and hence
                    \[
                        y = \frac{x^n}{(n - p)^2} + C_1 x^p \ln x + C_2 x^p.
                    \]
          \end{enumerate}
\end{enumerate}