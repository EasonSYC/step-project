\Question{\currfilebase}

\begin{enumerate}
    \item \begin{enumerate}
              \item We have
                    \[
                        \sqrt{4x^2 - 8x + 64} \leq \abs*{x + 8} \iff 0 \leq 4x^2 - 8x + 64 \leq (x + 8)^2.
                    \]

                    The left inequality can be simplified as follows:
                    \begin{align*}
                        4x^2 - 8x + 64 & \geq 0  \\
                        x^2 - 2x + 16  & \geq 0  \\
                        (X - 1)^2 + 15 & \geq 0,
                    \end{align*}
                    which is always true.

                    The right inequality can be simplified as follows:
                    \begin{align*}
                        4x^2 - 8x + 64 & \leq (x + 8)^2      \\
                        4x^2 - 8x + 64 & \leq x^2 + 16x + 64 \\
                        3x^2 - 24 x    & \leq 0              \\
                        x (x - 8)      & \leq 0,
                    \end{align*}
                    which solves to \(0 \leq x \leq 8\).

                    Hence, the solution to the original inequality is \(x \in [0, 8]\).

              \item WE have
                    \[
                        \sqrt{4x^2 - 8x + 64} \leq \abs*{3x - 8} \iff 0 \leq 4x^2 - 8x + 64 \leq (3x - 8)^2.
                    \]

                    The left inequality is always true from the previous part.

                    The right inequality can be simplified as follows:
                    \begin{align*}
                        4x^2 - 8x + 64 & \leq (3x - 8)^2      \\
                        4x^2 - 8x + 64 & \leq 9x^2 - 48x + 64 \\
                        5x^2 - 40x     & \geq 0               \\
                        x (x - 8)      & \geq 0,
                    \end{align*}
                    which solves to \(x \leq 0\) or \(x \geq 8\).

                    Hence, the solution to the original inequality is \(x \in (-\infty, 0] \cup [8, \infty)\).
          \end{enumerate}

    \item \begin{enumerate}
              \item We have
                    \begin{align*}
                        \left(\sqrt{4x^2 - 8x + 64} + 2(x - 1)\right)f(x) & = \left(\sqrt{4x^2 - 8x + 64}\right)^2 - [2 (x - 1)]^2      \\
                                                                          & = \left(4x^2 - 8x + 64\right) - 4 \left(x^2 - 2x + 1\right) \\
                                                                          & = \left(4x^2 - 8x + 64\right) - \left(4x^2 - 8x + 4\right)  \\
                                                                          & = 60.
                    \end{align*}

                    Hence,
                    \[
                        f(x) = \frac{60}{\sqrt{4x^2 - 8x + 64} + 2 (x - 1)}.
                    \]

                    As \(x \to \infty\), \(\sqrt{4x^2 - 8x + 64} \to \infty\), \(2 (x - 1) \to \infty\).

                    Hence, \(f(x) \to 0\) as \(x \to \infty\).

              \item Let \(f_1(x) = \sqrt{4x^2 - 8x + 64}\), \(f_2(x) = 2 (x - 1)\).

                    \(f_1(0) = \sqrt{64} = 8\), and \(f_2(0) = 2 \cdot (-1) = -2\).

                    We have \(f(x) = f_1(x) - f_2(x) > 0\) from the previous part, and that \(f_1(x)\) is defined for all \(x\) and is always positive.

                    Furthermore,
                    \[
                        f_1(x) = 2 \sqrt{x^2 - 2x + 16} = 2 \sqrt{(x - 1)^2 + 15},
                    \]
                    and hence \(f_1\) decreases on \((-\infty, 1)\) and increases on \((1, \infty)\), taking a minimum of \(f_1(1) = 2 \sqrt{15}\).

                    In terms of symmetry, we have \(f_1(1 - x) = f_1(1 + x)\) and \(f_2(1 - x) = - f_2(1 + x)\). \(f_2\) is an asymptote to \(f_1\) as \(x \to \infty\), and \(-f_2\) is an asymptote to \(f_1\) as \(x \to -\infty\).

                    Hence, the sketch looks as follows.

                    \begin{center}
                        \input{\currfiledir 2-diag1}
                    \end{center}
          \end{enumerate}

    \item Let \(x = 3\), and we must have \(\sqrt{4 \cdot 9 - 5 \cdot 3 + 4} = \abs*{3m + c}\), and hence \(5 = \abs*{3m + c}\).

          This is only achievable for \(m = \pm 2\) due to the diagram -- the solution set can only be 'one-sided' if on the other side the absolute value is eventually 'parallel' to the curve.

          We let \(m = 2\), and hence \(5 = \abs*{6 + c}\), which gives \(c = -1\) or \(c = -11\).

          We would like to show that the desired value is \(c = -1\), and that \(c = -11\) does not work.
          \[
              \sqrt{4x^2 - 5x + 4} \leq \abs*{2x - 1} \iff 0 \leq 4x^2 - 5x + 4 \leq (2x - 1)^2.
          \]

          The left inequality can be simplified as
          \[
              0 \leq 4x^2 - 5x + 4 = \left(2x - \frac{5}{4}\right)^2 + \frac{39}{16},
          \]
          and hence is always true.

          The right inequality can be simplified as
          \begin{align*}
              4x^2 - 5x + 4 & \leq (2x - 1)^2    \\
              4x^2 - 5x + 4 & \leq 4x^2 - 4x + 1 \\
              x             & \geq 3,
          \end{align*}
          and hence the solution set to the whole inequality is \(x \geq 3\) as desired.

          On the other hand, for the case of \(c = -11\), we have
          \[
              \sqrt{4x^2 - 5x + 4} \leq \abs*{2x - 11} \iff 0 \leq 4x^2 - 5x + 4 \leq (2x - 11)^2,
          \]
          and the left inequality is always true by previously. However, the right inequality simplifies as
          \begin{align*}
              4x^2 - 5x + 4 & \leq (2x - 11)^2      \\
              4x^2 - 5x + 4 & \leq 4x^2 - 44x + 121 \\
              39x           & \leq 117              \\
              x             & \leq 3,
          \end{align*}
          and the inequality is in the wrong direction.

          Hence, a possible value of \(m\) is \(2\), and the corresponding value of \(c\) is \(-1\).

    \item The diagram as follows shows the only possibility of the configuration.

          \begin{center}
              \input{\currfiledir 2-diag2}
          \end{center}

          Hence, we must have \(x^2 + px + q = mx + c\) for \(x = -5\) and \(x = 7\), and \(x^2 + px + q = -mx - c\) for \(x = 1\) and \(x = 5\).
          \[
              \left\{
              \begin{aligned}
                  25 - 5p + q & = -5m + c,    \\
                  49 + 7p + q & = 7m + c,     \\
                  1 + p + q   & = - (m + c),  \\
                  25 + 5p + q & = - (5m + c).
              \end{aligned}
              \right.
          \]

          Subtracting the first equation from the final equation gives \(10 p = -2c\), and hence \(c = -5p\).

          Subtracting the first equation from the second equation gives us \(24 + 12 p = 12 m\), and hence
          \(m = 2 + p\).

          Putting these into the third equation gives
          \begin{align*}
              q & = -m - c - ; - 1            \\
                & = - (2 + p) - (-5p) - p - 1 \\
                & = 3p - 3.
          \end{align*}

          Putting all these into the final equation gives
          \begin{align*}
              25 + 5p + (3p - 3) & = - \left[5 (2 + p) + (-5p)\right] \\
              25 + 8p - 3        & = - (10 + 5p - 5p)                 \\
              22 + 8p            & = -10                              \\
              8p                 & = -32                              \\
              p                  & = -4,
          \end{align*}
          and so \(q = -15, m = -2, c = 20\). Hence,
          \[
              (p, q, m, c) = (-4, -15, -2, 20).
          \]
\end{enumerate}