\Question{\currfilebase}

\begin{enumerate}
    \item \((y - x + 3) (y + x - 5) = 0\) if and only if \(y - x + 3 = 0\), or \(y + x - 5 = 0\). In the first case, \(y = x - 3\), representing a straight line with gradient \(1\), and in the second case, \(y = -x + 5\), representing a straight line with gradient \(-1\).

          The equation represents a pair of straight lines with gradients \(1\) and \(-1\) if and only if it could be factorised into the form \((y - x + a) (y + x - b)\).

          \begin{align*}
              (y - x + a) (y + x + b) & = y^2 + xy + by - xy - x^2 - bx + ay + ax + ab \\
                                      & = y^2 - x^2 + (a + b) y + (a - b) x + ab,
          \end{align*}
          and \(p = a + b, q = a - b, r = ab\).

          On one hand, if it could be factorised into this form, we have
          \[
              p^2 - q^2 = (a + b)^2 - (a - b)^2 = a^2 + 2ab + b^2 - a^2 + 2ab - b^2 = 4ab = 4r.
          \]

          On the other hand, let \(a = \frac{p + q}{2}\), \(b = \frac{p - q}{2}\), and we have
          \[
              a + b = p, a - b = q, ab = \frac{p + q}{2} \frac{p - q}{2} = \frac{p^2 - q^2}{4} = \frac{4r}{4} = r.
          \]

          This shows that this is a necessary and sufficient condition, which finishes our proof.

    \item Since the point \((x, y)\) lies on \(C_1\), we must have \(y = x^2\), and \(y - x^2 = 0\).

          Since it lies on \(C_2\), we must have \(x = y^2 + 2sy + s (s + 1)\), and \(y^2 + 2sy + s (s + 1) - x\).

          Hence,
          \begin{align*}
              \LHS & = y^2 + 2sy + s (s + 1) - x + k (y - x^2) \\
                   & = 0 + k \cdot 0                           \\
                   & = 0                                       \\
                   & = \RHS
          \end{align*}
          for any real number \(k\).

          Let \(k = 1\), by rearranging, we have
          \[
              y^2 - x^2 + (2s + 1) y - x + s (s + 1) = 0.
          \]

          We notice that
          \begin{align*}
              (2s + 1)^2 - (-1)^2 & = 4s^2 + 4s + 1 - 1 \\
                                  & = 4s^2 + 4s         \\
                                  & = 4 s (s + 1),
          \end{align*}
          which means that this represents a pair of straight lines with gradients \(1\) and \(-1\). The four points of intersection must lie on them.

    \item By part (ii), we have \(a = \frac{(2s + 1) - 1}{2} = s\), and \(b = \frac{(2s + 1) - (-1)}{2} = s + 1\). This means
          \[
              (y - x + s) (y + x + s + 1) = 0,
          \]
          and the lines are \(y = x - s\) and \(y = -x - s - 1\).

          Since a straight line may at most meet a polynomial twice, we must have \(y = x - s\) meets \(y = x^2\) at two distinct point, and \(y = -x - s - 1\) meets \(y = x^2\) at two distinct points as well.

          \(x^2 = x - s \iff x^2 - x + s = 0\), and hence \(1 - 4s > 0\), which shows that \(s < \frac{1}{4}\).

          \(x^2 - -x - s - 1 \iff x^2 + x + (s + 1) = 0\), and hence \(1 - 4(s + 1) > 0\), which shows that \(s < - \frac{3}{4}\).

          Hence, \(s < - \frac{3}{4}\).

    \item The lines are \(y = x - s\) and \(y = -x - s - 1\). Since \(s < - \frac{3}{4}\), both lines intersect \(y = x^2\) on precisely two points, since the discriminant for the quadratic is positive. Hence, we just have to show that none of those four points are the same.

          This could only be the case of the intersection of the intersection of the two lines, which is \(\left(-\frac{1}{2}, - \frac{2s + 1}{2}\right)\). This lies on \(y = x^2\) if and only if
          \[
              - \frac{2s + 1}{2} = \left(- \frac{1}{2}\right)^2 \iff -s - \frac{1}{2} = \frac{1}{4} \iff s =- \frac{3}{4}
          \]
          which is not the case here.

          Hence, \(C_1\) and \(C_2\) must intersect at four distinct points.
\end{enumerate}