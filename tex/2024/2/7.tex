\Question{\currfilebase}

\begin{enumerate}
    \item In this case, we have either \(y^2 + (x - 1)^2 = 1\) (giving a circle with radius \(1\) centred at \((1, 0)\)), or \(y^2 + (x + 1)^2 = 1\) (giving a circle with radius \(1\) centred at \((-1, 0)\)).
          \begin{center}
              \input{\currfiledir 7-diag1}
          \end{center}

    \item At \(y = k\), we have
          \begin{align*}
              [(x - 1)^2 + (k^2 - 1)] [(x + 1)^2 + (k^2 - 1)]                                      & = \frac{1}{16} \\
              (x - 1)^2 (x + 1)^2 + (k^2 - 1) [(x - 1)^2 + (x + 1)^2] + (k^2 - 1)^2 - \frac{1}{16} & = 0            \\
              (x^2 - 1)^2 + 2 (k^2 - 1) (x^2 + 1) + (k^4 - 2k^2 + 1) - \frac{1}{16}                & = 0            \\
              x^4 - 2x^2 + 1 + 2 (k^2 - 1) x^2 + 2 (k^2 - 1) + (k^4 - 2k^2 + 1) - \frac{1}{16}     & = 0            \\
              x^4 + 2 (k^2 - 2) x^2 + k^4 - \frac{1}{16}                                           & = 0,
          \end{align*}
          as desired.

          By completing the square, we have
          \begin{align*}
              (x^2 + (k^2 - 2))^2 + k^4 - \frac{1}{16} - (k^2 - 2)^2 & = 0                     \\
              (x^2 + (k^2 - 2))^2                                    & = \frac{65}{16} - 4k^2.
          \end{align*}

          \begin{itemize}
              \item If \(4k^2 > \frac{65}{16}\), i.e. \(k^2 > \frac{65}{64}\), the right-hand side is negative, so there will be no intersections.

              \item If \(4k^2 = \frac{65}{16}\), i.e. \(k^2 = \frac{65}{64}\), we have
                    \[
                        x^2 + (k^2 - 2) = 0,
                    \]
                    and hence
                    \[
                        x^2 = 2 - k^2 = 2 - \frac{65}{64} = \frac{63}{64},
                    \]
                    giving
                    \[
                        x = \pm \frac{3\sqrt{7}}{8}.
                    \]

                    There will be two intersections.

              \item If \(4k^2 < \frac{65}{16}\), i.e. \(k^2 < \frac{65}{64}\), we have
                    \[
                        x^2 + (k^2 - 2) = \pm \sqrt{\frac{65}{16} - 4k^2},
                    \]
                    and hence
                    \[
                        x^2 = 2 - k^2 \pm \sqrt{\frac{65}{16} - 4k^2}.
                    \]

                    The case where
                    \begin{align*}
                        x^2 & = 2 - k^2 + \sqrt{\frac{65}{16} - 4k^2} \\
                            & > 2 - k^2                               \\
                            & > 2 - \frac{65}{64}                     \\
                            & = \frac{63}{64}                         \\
                            & > 0
                    \end{align*}
                    always gives two solutions for \(x\).

                    \begin{itemize}
                        \item If \(2 - k^2 - \sqrt{\frac{65}{16} - 4k^2} < 0\),
                              \begin{align*}
                                  2 - k^2 - \sqrt{\frac{65}{16} - 4k^2} & < 0              \\
                                  \sqrt{\frac{65}{16} - 4k^2}           & > 2 - k^2        \\
                                  \frac{65}{16} - 4k^2                  & > k^4 - 4k^2 + 4 \\
                                  k^4                                   & < \frac{1}{16}   \\
                                  k^2                                   & < \frac{1}{4},
                              \end{align*}
                              there are no solutions for the case where the minus sign is taken.
                        \item If \(2 - k^2 - \sqrt{\frac{65}{16} - 4k^2} = 0\), \(k^2 = \frac{1}{4}\), the minus sign produce precisely one solution \(x = 0\), giving \(3\) intersections in total.
                        \item If \(2 - k^2 - \sqrt{\frac{65}{16} - 4k^2} < 0\), \(k^2 > \frac{1}{4}\), the minus sign will produce two additional roots, hence giving \(4\) intersections in total.
                    \end{itemize}
          \end{itemize}

          To summarise, the number of intersections with the line \(y = k\) for each positive value of \(k\) is
          \[
              \text{number of intersections} = \begin{cases}
                  0, & k^2 > \frac{65}{64}, k > \frac{\sqrt{65}}{8},                             \\
                  2, & k^2 = \frac{65}{64}, k = \frac{\sqrt{65}}{8},                             \\
                  4, & \frac{1}{4} < k^2 < \frac{65}{64}, \frac{1}{2} < k < \frac{\sqrt{65}}{8}, \\
                  3, & k^2 = \frac{1}{4}, k = \frac{1}{2},                                       \\
                  2, & k^2 < \frac{1}{4}, 0 < k < \frac{1}{2}.
              \end{cases}
          \]

    \item When the point on \(C_2\) has the greatest possible \(y\)-coordinate, the two points have \(x\)-coordinates
          \[
              x = \pm \frac{3\sqrt{7}}{8},
          \]
          and on \(C_1\) has
          \[
              x = \pm 1.
          \]

          Since \(3 \sqrt{7} = \sqrt{63} < \sqrt{64} = 8\), we must have \(\frac{3\sqrt{7}}{8} < 1\), meaning those on \(C_2\) are closer to the \(y\)-axis than those on \(C_1\).

    \item If both are negative, then the distance from \((x, y)\) to \((1, 0)\) and \((-1, 0)\) are both less than \(1\). But this is not possible, since the distance from \((1, 0)\) to \((-1, 0)\) is \(2\), which means the sum of the distances from \((x, y)\) to those points has to be at least \(2\).

          Therefore, since the product of those two terms are positive for \(C_2\), and they cannot be both negative, they must both be positive, and hence the distance from \((x, y)\) to \((1, 0)\) and \((-1, 0)\) are both more than \(1\), meaning all points on \(C_2\) lies outside the two circles that make up \(C_1\), which shows that \(C_2\) lies entirely outside \(C_1\).

    \item \(C_2\) is symmetric about both the \(x\)-axis and the \(y\)-axis.

          When \(x = 0\), \(y^4 = \frac{1}{16}\), and hence \(y = \pm \frac{1}{2}\).

          When \(y = 0\), \(x^2 = 2 + \frac{\sqrt{65}}{16}\), and hence \(x = \pm \sqrt{2 + \frac{\sqrt{65}}{{4}}} = \pm \frac{1}{2} \sqrt{8 + \sqrt{65}}\).

          Hence, the graph looks as follows.
          \begin{center}
              \input{\currfiledir 7-diag2}
          \end{center}
\end{enumerate}