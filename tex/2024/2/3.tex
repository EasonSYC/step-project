\Question{\currfilebase}

\begin{enumerate}
    \item The line \(NP\) has gradient
          \[
              m_{NP} = \frac{\sin \theta - 0}{\cos \theta - (-1)} = \frac{\sin \theta}{\cos \theta + 1},
          \]
          and hence it has equation
          \[
              l_{NP}: y = \frac{\sin \theta}{\cos \theta + 1} \cdot (x + 1).
          \]

          When \(x = 0\), we have
          \begin{align*}
              q & = \frac{\sin \theta}{\cos \theta + 1}                                                     \\
                & = \frac{2 \sin \frac{\theta}{2} \cos \frac{\theta}{2}}{2 \cos^2 \frac{\theta}{2} - 1 + 1} \\
                & = \frac{\sin \frac{\theta}{2}}{\cos \frac{\theta}{2}}                                     \\
                & = \tan \frac{\theta}{2}.
          \end{align*}

    \item \begin{enumerate}
              \item \begin{align*}
                        \RHS & = \tan \frac{1}{2} \left(\theta + \frac{1}{2} \pi\right)                                          \\
                             & = \tan \left(\frac{\theta}{2} + \frac{\pi}{4}\right)                                              \\
                             & = \frac{\tan \frac{\theta}{2} + \tan \frac{\pi}{4}}{1 - \tan \frac{\theta}{2} \tan \frac{\pi}{4}} \\
                             & = \frac{q + 1}{1 - q}                                                                             \\
                             & = f_1(q),
                    \end{align*}
                    as desired.

              \item Let the coordinates of \(P_1\) be \((\cos \phi, \sin \phi)\), and hence we must have
                    \begin{align*}
                        f_1(q)                                                & = \tan \frac{1}{2} \phi    \\
                        \tan \frac{1}{2} \left(\theta + \frac{1}{2}\pi\right) & = \tan \frac{1}{2} \phi    \\
                        \phi                                                  & = \theta + \frac{1}{2}\pi,
                    \end{align*}
                    and so \(P_1\) is the image of \(P\) being rotated through an angle of \(\pi\) counterclockwise about the origin.
          \end{enumerate}

    \item \begin{enumerate}
              \item The coordinates of \(P_2\) are \(\left(\cos\left(\theta + \frac{1}{3}\pi\right), \sin\left(\theta + \frac{1}{3}\pi\right)\right)\), and hence we must have that
                    \begin{align*}
                        f_3(q) & = \tan \frac{1}{2} \left(\theta + \frac{1}{3}\pi\right)                                           \\
                               & = \tan \left(\frac{\theta}{2} + \frac{\pi}{6}\right)                                              \\
                               & = \frac{\tan \frac{\theta}{2} + \tan \frac{\pi}{6}}{1 - \tan \frac{\theta}{2} \tan \frac{\pi}{6}} \\
                               & = \frac{q + \frac{1}{\sqrt{3}}}{1 - q \cdot \frac{1}{\sqrt{3}}}                                   \\
                               & = \frac{1 + \sqrt{3}q}{\sqrt{3} - q}.
                    \end{align*}

              \item Notice that \(f_3(q) = f_1(-q) = \tan \frac{1}{2} \left(- \theta + \frac{1}{2}\pi\right)\), and so the coordinates of \(P_3\) must be
                    \[
                        \left(\cos \left(\frac{1}{2}\pi - \theta\right), \sin \left(\frac{1}{2}\pi - \theta\right)\right),
                    \]
                    which is \(P_3 (\sin \theta, \cos \theta)\), a reflection of \(P\) in the line \(y = x\).

              \item \(P_4\) must be the image of \(P\) under the following transformations:
                    \begin{itemize}
                        \item Rotation counterclockwise by \(\frac{1}{3}\pi\) about the origin \(O\);
                        \item Reflection in the line \(y = x\);
                        \item Rotation clockwise by \(\frac{1}{3}\pi\) about the origin \(O\).
                    \end{itemize}

                    This is precisely the reflection in which the axis after the second step is \(y = x\). Hence, the axis of this reflection has an angle of \(\frac{1}{4} \pi - \frac{1}{3}\pi = \frac{1}{12}\pi\) with the positive \(x\)-axis.

                    \(P_4\) is the image of \(P\) reflected in the line which makes an angle of \(-\frac{\pi}{12}\) with the positive \(x\)-axis, passing through the origin.
          \end{enumerate}
\end{enumerate}