\Question{\currfilebase}
\begin{enumerate}
    \item Notice that
          \begin{align*}
              z & = \frac{\exp(i\theta) + \exp(i\phi)}{\exp(i\theta) - \exp(i\phi)}                                                                           \\
                & = \frac{\exp(i\theta) + \exp(i\phi)}{\exp(i\theta) - \exp(i\phi)} \cdot \frac{\exp(-i\theta) - \exp(-i\phi)}{\exp(-i\theta) - \exp(-i\phi)} \\
                & = \frac{1 + \exp(i\phi - i\theta) - \exp(i\theta - i\phi) - 1}{1 - \exp(i \theta - i \phi) - \exp(i \phi - i \theta) + 1}                   \\
                & = \frac{\exp(i(\phi - \theta)) - \exp(-i(\phi - \theta))}{2 - \exp(- i (\phi - \theta)) - \exp(i (\phi - \theta))}                          \\
                & = \frac{2i \sin (\phi - \theta)}{2 - 2 \cos (\phi - \theta)}                                                                                \\
                & = \frac{i\sin (\phi - \theta)}{1 - \cos (\phi - \theta)}                                                                                    \\
                & = \frac{i \cdot 2 \sin \frac{\phi - \theta}{2} \cos \frac{\phi - \theta}{2}}{1 - (1 - 2 \sin^2 \frac{\phi - \theta}{2})}                    \\
                & = \frac{2i \sin \frac{\phi - \theta}{2} \cos \frac{\phi - \theta}{2}}{2 \sin^2 \frac{\phi - \theta}{2}}                                     \\
                & = i \cot \frac{\phi - \theta}{2},
          \end{align*}
          as desired.

          The modulus of \(z\) is \(\abs*{\cot \frac{\phi - \theta}{2}}\). The argument of \(z\) is \(\pm \frac{\pi}{2}\).

    \item Let \(a = \exp(i \alpha)\), and \(b = \exp(i \beta)\), where \(a - b \neq 2n\pi\) for integer \(n\) (this ensures that \(A\) and \(B\) are distinct). We must have \(x = a + b = \exp(i\alpha) + \exp (i \beta)\), and \(b - a = \exp(i \beta) - \exp(i \alpha)\).

          The vectors representing the two complex numbers are perpendicular, if and only if their argument differ by \(\pm \frac{\pi}{2}\), if and only if their ratio has argument \(\pm \frac{\pi}{2}\). Notice that the ratios
          \begin{align*}
              \frac{OX}{AB} & = \frac{a + b}{b - a}                                                 \\
                            & = \frac{\exp (i\alpha) + \exp(i\beta)}{\exp(i\beta) - \exp(i \alpha)}
          \end{align*}
          takes the same form as \(z\) before, and hence has argument \(\pm \frac{\pi}{2}\). This hence means \(OX\) is perpendicular to \(AB\).

    \item Similarly, let \(a = \exp(i \alpha)\), \(b = \exp(i \beta)\), and \(c = \exp(i \gamma)\), where no pair of \(\alpha, \beta\) and \(\gamma\) differ by some multiple of \(2\pi\) (which ensures that \(A, B, C\) are distinct points).

          If \(H\) is the orthocentre of triangle \(ABC\), then
          \[
              h = a + b + c = \exp (i \alpha) + \exp (i \beta) + \exp (i \gamma),
          \]
          and hence
          \[
              AH = h - a = b + c = \exp (i \beta) + \exp (i \gamma),
          \]
          \[
              BC = c - b = \exp (i \gamma) - \exp (i \beta).
          \]

          If \(h \neq a\), then \(AH = b + c \neq 0\), then the angle between \(AH\) and \(BC\) is given by the argument of the ratio of the complex numbers representing them, and notice
          \[
              \frac{AH}{BC} = \frac{\exp (i \beta) + \exp (i \gamma)}{\exp (i \gamma) - \exp (i \beta)},
          \]
          which takes the same form of \(z\) in the first part. Hence, the argument of this must be \(\pm \frac{\pi}{2}\) since \(b + c \neq 0\), which shows that \(AH\) is perpendicular to \(BC\).

          This means that either \(h = a\), or \(AH\) is perpendicular to \(BC\), as desired.

    \item Similarly, let \(a = \exp(i \alpha)\), \(b = \exp(i \beta)\), \(c = \exp(i \gamma)\) and \(d = \exp(i \delta)\), where no pair of \(\alpha\), \(\beta\), \(\gamma\) and \(\delta\) differ by some multiple of \(2\pi\) (which ensures that \(A, B, C, D\) are distinct points). Hence,
          \[
              q = b + c + d = \exp(i \beta) + \exp(i \gamma) + \exp(i \delta),
          \]
          and the midpoint of \(AQ\), \(M\), represented by complex number \(m\), is given by
          \[
              m = \frac{a + b + c + d}{2}.
          \]

          By symmetry, the midpoint of \(BR\), \(CS\) and \(DP\) must also be \(M\).

          This means that by an enlargement of scale factor \(-1\) about \(M\), \(A\) will be transformed to \(Q\), \(B\) to \(R\), \(C\) to \(S\), and \(D\) to \(P\).

          Hence, \(ABCD\) is transformed to \(PQRS\) by an enlargement of scale factor \(-1\), with centre of enlargement being \(\frac{a + b + c + d}{4}\), the midpoint of \(AQ\).
\end{enumerate}