\Question{\currfilebase}

\begin{enumerate}
    \item Let
          \[
              f(t) = \frac{2t}{1 - t^2}.
          \]

          By the double angle formula for \(\tan\), we have
          \[
              f(\tan \theta) = \tan 2\theta.
          \]

          Since \(y = f(x)\), we have \(y = f(\tan \alpha) = \tan 2 \alpha\). Similarly, \(z = f(y) = \tan 4 \alpha\), and \(x = f(z) = \tan 8 \alpha\).

          But since \(x = x\), we must have \(\tan \alpha = \tan 8 \alpha\), and there must be some \(k \in \ZZ\) such that
          \[
              \alpha + k\pi = 8 \alpha,
          \]
          i.e.
          \[
              \alpha = \frac{k\pi}{7}.
          \]

          Since \(-\frac{1}{2}\pi < \alpha < \frac{1}{2}\pi\) for the substitution, we have
          \[
              \alpha = -\frac{3}{7}\pi, -\frac{2}{7}\pi, -\frac{1}{7}\pi, 0, \frac{1}{7}\pi, \frac{2}{7}\pi, \frac{3}{7}\pi,
          \]
          and hence
          \[
              (\alpha, 2\alpha, 4\alpha) = \left(0, 0, 0\right), \left(\pm \frac{1}{7}\pi, \pm \frac{2}{7}\pi, \pm \frac{4}{7}\pi\right), \left(\pm \frac{2}{7}\pi, \pm \frac{4}{7}\pi, \pm \frac{8}{7}\pi\right), \left(\pm \frac{3}{7}\pi, \pm \frac{6}{7}\pi, \pm \frac{12}{7}\pi\right),
          \]
          which means
          \[
              (x, y, z) = \left(\tan 0, \tan 0, \tan 0\right),
          \]
          or
          \[
              (x, y, z) = \left(\tan \pm \frac{1}{7} \pi, \tan \pm \frac{2}{7}\pi, \tan \pm \frac{4}{7}\pi\right) = \left(\tan \pm \frac{1}{7} \pi, \tan \pm \frac{2}{7}\pi, \tan \mp \frac{3}{7}\pi\right),
          \]
          or
          \[
              (x, y, z) = \left(\tan \pm \frac{2}{7}\pi, \tan \pm \frac{4}{7}\pi, \tan \pm \frac{8}{7}\pi\right) = \left(\tan \pm \frac{2}{7}\pi, \tan \mp \frac{3}{7}\pi, \tan \pm \frac{1}{7}\pi\right),
          \]
          or
          \[
              (x, y, z) = \left(\tan \pm \frac{3}{7}\pi, \tan \pm \frac{6}{7}\pi, \tan \pm \frac{12}{7}\pi\right) = \left(\tan \pm \frac{3}{7}\pi, \tan \mp \frac{1}{7}\pi, \tan \mp \frac{2}{7}\pi\right).
          \]

    \item Let
          \[
              g(t) = \frac{3t - t^3}{1 - 3t^2}.
          \]

          The triple angle formula for \(\tan\) is given by
          \begin{align*}
              \tan 3\theta & = \frac{\tan \theta + \tan 2\theta}{1 - \tan \theta \tan 2\theta}                                                       \\
                           & = \frac{\tan \theta + \frac{2 \tan \theta}{1 - \tan^2 \theta}}{1 - \tan \theta \frac{2 \tan \theta}{1 - \tan^2 \theta}}
                           & = \frac{\tan \theta (1 - \tan^2 \theta) + 2 \tan \theta}{(1 - \tan^2 \theta) - \tan \theta (2 \tan \theta)}             \\
                           & = \frac{3 \tan \theta - \tan^3 \theta}{1 - 3 \tan^2 \theta},
          \end{align*}
          and hence
          \[
              g (\tan \theta) = \tan 3 \theta.
          \]

          Let \(x = \tan \alpha\) for \(-\frac{\pi}{2} < \alpha < \frac{\pi}{2}\). We must have \(y = \tan 3\alpha\), \(z = \tan 9 \alpha\) and \(x = \tan 27 \alpha\).

          There must exist some \(k \in \ZZ\) such that
          \[
              \alpha + k \pi = 27 \alpha,
          \]
          and hence
          \[
              \alpha = \frac{k\pi}{26}.
          \]

          It must be the case that \(-13 < k < 13\), and this leads to \(-12 \leq k \leq 12\). These all lead to distinct values of \(x\).

          We already have \(\alpha \neq t\pi + \frac{\pi}{2}\) for any \(t \in \ZZ\).

          We still verify that \(2\alpha \neq t\pi + \frac{\pi}{2}\). We have that
          \begin{align*}
              2 \alpha - \frac{\pi}{2} & = \frac{k\pi}{13} - \frac{\pi}{2} \\
                                       & = \frac{(2k - 13)\pi}{26}.
          \end{align*}

          \(2k - 13\) cannot be a multiple of \(13\) apart from \(k = 0\) (in which case it is still not a multiple of \(26\)), hence not of \(26\), and hence \(2\alpha \neq t\pi + \frac{\pi}{2}\).

          A similar reasoning applies for \(4\alpha\):
          \begin{align*}
              4 \alpha - \frac{\pi}{2} & = \frac{2k\pi}{13} - \frac{\pi}{2} \\
                                       & = \frac{(4k - 13)\pi}{26}.
          \end{align*}

          \(4k - 13\) cannot be a multiple of \(13\) apart from \(k = 0\) (in which case it is still not a multiple of \(26\)), hence not of \(26\), and hence \(4\alpha \neq t\pi + \frac{\pi}{2}\).

          Therefore, all \(25\) values of \(k\) leads to pairs of solutions for \((x, y, z)\), and they must all be distinct (since \(x\)s) are distinct.

          Therefore, there are \(25\) pairs of distinct real solutions to the simultaneous solutions.

    \item \begin{enumerate}
              \item Let \(h(t) = 2t^2 - 1\). Notice that by the cosine double angle formula,
                    \[
                        h(\cos \theta) = \cos 2\theta.
                    \]

                    If \(\abs*{x}, \abs*{y}, \abs*{z} \leq 1\), let \(x = \cos \alpha\) for \(0 \leq \alpha \leq \pi\). We must have \(y = \cos 2\alpha, z = \cos 4\alpha\), and \(x = \cos 8 \alpha\), leading to \(\cos \alpha = \cos 8 \alpha\).

                    Hence, we must have, for \(k \in \ZZ\), that
                    \[
                        8\alpha = 2k\pi \pm \alpha,
                    \]
                    which gives
                    \[
                        \alpha = \frac{2k\pi}{7}
                    \]
                    or
                    \[
                        \alpha = \frac{2k\pi}{9}.
                    \]

                    Therefore, we have
                    \[
                        \alpha = 0, \frac{2\pi}{7}, \frac{4\pi}{7}, \frac{6\pi}{7}, \frac{2\pi}{9}, \frac{4\pi}{9}, \frac{6\pi}{9}, \frac{8\pi}{9}
                    \]
                    which gives \(8\) pairs of solutions for \((x, y, z)\).

              \item We have \(x = h^3(x)\), and hence \(x\) satisfies a polynomial with degree \(8\). Hence, there are at most \(8\) distinct real roots for \(x\), and since there are \(8\) of them for which \(\abs*{x} \leq 1\), it must be the case that they are all of them. Hence, all solutions to the equations satisfy \(\abs*{x}, \abs*{y}, \abs*{z} \leq 1\).
          \end{enumerate}
\end{enumerate}