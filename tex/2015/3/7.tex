\Question{\currfilebase}

Note that
\begin{align*}
    D^2 x^a & = x \DiffOp{x} \left(x \DiffOp{x} x^a\right)   \\
            & = x \DiffOp{x} \left(x \cdot ax^{a - 1}\right) \\
            & = ax \DiffOp{x} x^a                            \\
            & = ax \cdot a \cdot x^{a - 1}                   \\
            & = a^2 x^a,
\end{align*}
as desired.

\begin{enumerate}
    \item Since we have that \(\Diff x^a = x \cdot x^{a - 1} \cdot a = a x^a\), and \(a\) is just a constant, then we must have
          \[
              D^n x^a = a^n x^a.
          \]

          If \(P(x)\) is a polynomial of degree \(r\), let
          \[
              P(x) = \sum_{k = 0}^{r} t_k x^k.
          \]

          Therefore,
          \[
              D^n P(x) = \sum_{k = 0}^{r} k^n t_k x^k.
          \]

          Notice that the highest degree term is \(r^n t_r x^r\).

          Since \(P(x)\) originally has degree \(r \geq 1\), we have \(r \neq 0\) and \(t_r \neq 0\), and therefore this term is non-zero.

          This implies \(D^n P(x)\) has degree \(r\) as well.

    \item We show this by induction on \(n\). The base case where \(n = 0\) is trivially true if we define \(D^0\) as the identity. Now, assume this is true for some \(n = k < m - 1\), i.e.
          \[
              D^{k} (1 - x)^m = (1 - x)^{m - k} \cdot Q(x)
          \]
          for some polynomial \(Q\), we aim to show this for \(n = k + 1 < m\). We have
          \begin{align*}
              D^{k + 1} (1 - x)^m & = D \left[(1 - x)^{m - k} \cdot Q(x)\right]                                \\
                                  & = x \left[-(m - k) (1 - x)^{m - k - 1} Q(x) + (1 - x)^{m - k} Q'(x)\right] \\
                                  & = (1 - x)^{m - k - 1} x\left[- (m - k) Q(x) + (1 - x) Q'(x)\right],
          \end{align*}
          which shows \(D^{k + 1} (1 - x)^m\) is divisible by \((1 - x)^{m - k - 1}\) which finishes our induction step. Hence, by the principle of mathematical induction, the original statement holds for any \(n < m\).

    \item Notice that
          \[
              (1 - x)^m = \sum_{r = 0}^{m} \binom{m}{r} (-x)^r,
          \]
          and hence
          \[
              D^n (1 - x)^m = \sum_{r = 0}^{m} (-1)^r \binom{m}{r} r^n x^r.
          \]

          Evaluate this at \(x = 1\), we can see
          \[
              \SqEvalAt{D^n (1 - x)^m}{x = 1} = \sum_{r = 0}^{m} (-1)^r \binom{m}{r} r^n 1^r =  \sum_{r = 0}^{m} (-1)^r \binom{m}{r} r^n.
          \]

          But for \(n < m\), \(D^n (1 - x)^m\) is divisible by \((1 - x)^{m - n}\) and hence by \((1 - x)\). This means that
          \[
              \SqEvalAt{D^n (1 - x)^m}{x = 1} = 0.
          \]

          Hence,
          \[
              \sum_{r = 0}^{m} (-1)^r \binom{m}{r} r^n = 0,
          \]
          as desired.
\end{enumerate}