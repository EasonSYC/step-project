\Question{\currfilebase}

\begin{enumerate}
    \item First, notice that
          \[
              \DiffFrac{y}{x} = \frac{\Diff{y} / \Diff{\theta}}{\Diff{x} / \Diff{\theta}} = \frac{\Diff{r} / \Diff{\theta} \cdot \sin \theta + r \cdot \cos \theta}{\Diff{r} / \Diff{\theta} \cdot \cos \theta - r \cdot \sin \theta}.
          \]

          Therefore, the original differential equation reduces to
          \[
              (r \sin \theta + r \cos \theta) \frac{\Diff{r} / \Diff{\theta} \cdot \sin \theta + r \cdot \cos \theta}{\Diff{r} / \Diff{\theta} \cdot \cos \theta - r \cdot \sin \theta} = r \sin \theta - r \cos \theta
          \]
          which further reduces to (since \(r \neq 0\))
          \[
              (\sin \theta + \cos \theta) \left[\DiffFrac{r}{\theta} \cdot \sin \theta + r \cos \theta\right] = (\sin \theta - \cos \theta) \left[\DiffFrac{r}{\theta} \cdot \cos \theta - r \sin \theta\right].
          \]

          Expanding the brackets and cancelling the equivalent terms gives us
          \[
              r \cos^2 \theta + \DiffFrac{r}{\theta}\sin^2\theta = - \DiffFrac{r}{\theta} \cos^2 \theta - r \sin^2 \theta,
          \]
          which reduces to (due to the Pythagoras Theorem \(\sin^2 \theta + \cos^2 \theta = 1\)),
          \[
              \DiffFrac{r}{\theta} + r = 0,
          \]
          as desired.

          The rearrangement (since \(r \neq 0\))
          \[
              \frac{\Diff{r}}{r} = - \Diff{\theta}
          \]
          shows that the solution to this differential equation must satisfy that (since \(r > 0\))
          \[
              \ln r = - \theta + C,
          \]
          i.e.
          \[
              r = A \exp (- \theta),
          \]
          where \(A > 0\).

          For critical values, notice that when \(\theta = 0\), \(r = A\), and when \(\theta = 2\pi\), \(r = \frac{A}{\exp 2\pi}\), and that \(r\) is decreasing with \(\theta\). The graph will look like a spiral

          A sketch is shown below, for \(\theta \in [0, 2\pi)\).

          \begin{center}
              \input{\currfiledir 8-diag1.tex}
          \end{center}

    \item Similar to the previous part, the equation reduces to
          \[
              \left(\sin \theta + \cos \theta - \cos \theta \cdot r^2\right) \left[\DiffFrac{r}{\theta} \cdot \sin \theta + r \cos \theta\right] = \left(\sin \theta - \cos \theta - \sin \theta \cdot r^2\right) \left[\DiffFrac{r}{\theta} \cdot \cos \theta - r \sin \theta\right],
          \]
          and hence, by expanding brackets and eliminating terms,
          \[
              \DiffFrac{r}{\theta} \sin^2 \theta + r \cos^2 \theta - r^3 \cos^2 \theta = -r \sin^2 \theta - \DiffFrac{r}{\theta} \cos^2\theta + r^3 \sin^2 \theta,
          \]
          which then simplifies to
          \[
              \DiffFrac{r}{\theta} + r - r^3 = 0.
          \]

          Notice that \(r = 1\) is a solution to this differential equation. Therefore, rearranging terms, we have
          \[
              \frac{\Diff{r}}{r^3 - r} = \Diff{\theta}.
          \]

          By partial fractions
          \[
              \frac{1}{r^3 - r} = -\frac{1}{r} + \frac{1}{2(r + 1)} + \frac{1}{2(r - 1)},
          \]
          we therefore must have
          \[
              \left[-\frac{1}{r} + \frac{1}{2(r + 1)} + \frac{1}{2(r - 1)}\right] \cdot \Diff{r} = \Diff \theta.
          \]

          This therefore means that
          \[
              \frac{1}{2} \ln \abs*{r + 1} + \frac{1}{2} \ln \abs*{r - 1} - \ln \abs*{r} = \theta + C,
          \]
          for some constant \(C \in \RR\).

          Combining logarithms and absolute values gives us
          \[
              \ln \abs*{\frac{r^2 - 1}{r^2}} = 2\theta + C,
          \]
          and therefore,
          \[
              \frac{r^2 - 1}{r^2} = \pm \exp C \cdot \exp(2\theta),
          \]
          and this can be simplified to
          \[
              1 - \frac{1}{r^2} = \pm \exp C \cdot \exp(2\theta),
          \]
          and therefore
          \[
              r^2 = \frac{1}{1 \mp \exp C \cdot \exp(2\theta)}.
          \]

          Let \(A = \mp \exp C \neq 0\), and therefore
          \[
              r^2 = \frac{1}{1 + A \exp(2 \theta)}.
          \]

          Notice when \(r = 1\), \(r\) satisfies that
          \[
              r^2 = 1,
          \]
          so the general solution will be
          \[
              r^2 = \frac{1}{1 + A \exp(2 \theta)}
          \]
          for \(A \in \RR\) which this equation makes sense.

          We restrict ourselves to \(\theta \in [0, 2\pi)\).

          Notice that, this equation makes sense for all \(A \geq 0\), since the denominator is obviously non-negative.

          For \(A < 0\), the denominator is decreasing in \(\theta\), and we would like it to be greater than zero for some \(\theta \in [0, 2\pi)\). Therefore, we would like the maximum possible value of the denominator to be greater than, that is when \(\theta = 0\):
          \[
              1 + A \exp 0 > 0,
          \]
          which gives \(A > -1\).

          We consider three cases where \(r > 0\), i.e.,
          \[
              r = \frac{1}{\sqrt{1 + A \exp(2\theta)}}.
          \]

          Notice this always passes through \(\left(\frac{1}{\sqrt{1 + A}}, 0\right)\).

          \begin{itemize}
              \item When \(-1 < A < 0\), the curve is not defined for
                    \[
                        1 + A \exp (2\theta) \leq 0,
                    \]
                    and this is precisely when
                    \[
                        \exp 2\theta \geq -\frac{1}{A},
                    \]
                    which is
                    \[
                        \theta \geq \frac{1}{2} \cdot \ln\left(-\frac{1}{A}\right).
                    \]

                    This means the curve will have an asymptote of line
                    \[
                        \theta = \frac{1}{2} \cdot \ln\left(-\frac{1}{A}\right).
                    \]

                    Also note that \(r\) is increasing in \(\theta\) in this case, and \(r \to \infty\) as \(\theta \to\) the asymptote.

                    \begin{center}
                        \input{\currfiledir 8-diag2.tex}
                    \end{center}

              \item When \(A = 0\), notice this just gives \(r = 1\), which is a circle with radius 1 centred at the origin.

                    \begin{center}
                        \input{\currfiledir 8-diag3.tex}
                    \end{center}

              \item In the final case where \(A > 0\), the following case arises.

                    \begin{center}
                        \input{\currfiledir 8-diag4.tex}
                    \end{center}
          \end{itemize}
\end{enumerate}