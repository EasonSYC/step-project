\Question{\currfilebase}

\begin{enumerate}
    \item \begin{itemize}
              \item \textbf{Only-if direction.} If \(w, z\) are real, then \(u = w + z\) and \(v = w^2 + z^2\) are real. Also,
                    \begin{align*}
                        2v - u^2 & = 2(w^2 + z^2) - (w + z)^2 \\
                                 & = w^2 - 2wz + z^2          \\
                                 & = (w - z)^2                \\
                                 & \geq 0,
                    \end{align*}
                    which implies \(u^2 \leq 2v\) as desired.

              \item \textbf{If direction.} If \(u, v \in \RR\) and \(u^2 \leq 2v\), we notice that
                    \[
                        wz = \frac{u^2 - v}{2} \in \RR.
                    \]

                    Hence, \(w, z\) are solutions to the quadratic equation
                    \[
                        x^2 - ux + \frac{u^2 - v}{2} = 0.
                    \]

                    Notice all coefficients in this equation is real. The discriminant satisfies
                    \begin{align*}
                        \Delta & = (-u)^2 - 4 \cdot 1 \cdot \frac{u^2 - v}{2} \\
                               & = u^2 - 2(u^2 - v)                           \\
                               & = 2v - u^2                                   \\
                               & \geq 0,
                    \end{align*}
                    which implies both solutions must be real, i.e. \(w, z\) are real, as desired.
          \end{itemize}

    \item By simplification, we notice that letting \(u = w + z\) and \(v = w^2 + z^2\), we have
          \begin{align*}
              w^3 + z^3 & = (w + z)(w^2 + z^2) - wz (w + z)                                   \\
                        & = (w + z)(w^2 + z^2) - \frac{1}{2} ((w + z)^2 - (w^2 + z^2))(w + z) \\
                        & = uv - \frac{u(u^2 - v)}{2}                                         \\
                        & = u \left(v - \frac{u^2 - v}{2}\right)                              \\
                        & = \frac{u}{2} \left(2v - (u^2 - v)\right)                           \\
                        & = \frac{u (3v - u^2)}{2}.
          \end{align*}

          This means,
          \[
              -\lambda + \lambda u = \frac{u \left[3 \cdot \left(u^2 - \frac{2}{3}\right) - u^2\right]}{2}
          \]
          which simplifies to
          \[
              (u - 1)(u^2 + u - \lambda) = 0.
          \]

          Therefore, \(u_1 = 1\). The discriminant of the remaining quadratic is
          \[
              \Delta = 1 + 4\lambda > 1 > 0,
          \]
          since \(\lambda > 0\).

          Therefore, \(u\) must always be real.

          The only case where there are less than \(3\) possible values of \(u\), is when \(u_1 = 1\) is also a solution to the quadratic.

          This is precisely when \(\lambda = u^2 + u = 1^2 + 1 = 2\).

          Apart from this case, the two real solutions to the quadratic are distinct and must not be \(1\), and there are three real values of \(u\).

          Since \(u\) is always real, \(u = w + z\) is always real and \(v = w^2 + z^2\) is always real. However, notice that
          \[
              2v - u^2 = 2 \cdot \left(u^2 - \frac{2}{3}\right) - u^2 = u^2 - \frac{4}{3}.
          \]

          But when \(u = 1\), \(2v - u^2 < 0\), \(2v < u^2\), and by part (i) at least one of \(w, z\) is not real.
\end{enumerate}